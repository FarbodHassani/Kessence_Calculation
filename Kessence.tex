%%%%%%%%%%%%%%%%%%%%%%%%%%
%   DRAFT - DEFLECTION ANGLES 
%%%%%%%%%%%%%%%%%%%%%%%%%%

\documentclass[a4paper,11pt]{article}
\pdfoutput=1
\usepackage{jcappub}
\usepackage{bbold}

\usepackage[english]{babel}
\usepackage[utf8]{inputenc}
\usepackage{amsmath}
\usepackage{color}
\usepackage{amsfonts}
\usepackage{graphicx}
\usepackage{amssymb}
\usepackage{eufrak}
\usepackage{etoolbox}
\usepackage{amsmath}
\usepackage{empheq}
\usepackage{cancel}
\usepackage[most]{tcolorbox}
\newtcbox{\mymath}[1][]{%
    nobeforeafter, math upper, tcbox raise base,
    enhanced, colframe=yellow!30!black,
    colback=yellow!30, boxrule=1pt,
    #1}
%%%%%%%%%%%%%%%%%%%%%%%%%%%%

\def\be{\begin{equation}}
\def\ee{\end{equation}}
\def\bea{\begin{eqnarray}}
\def\eea{\end{eqnarray}}
\def\bean{\begin{eqnarray*}}
\def\eean{\end{eqnarray*}}
\def\cd{\cdot}
\def\vp{\varphi}
\def\l {\langle}
\def\re {\rangle}
\def \dd {\partial}
\def \ra {\rightarrow}
\def \la {\lambda}
\def \La {\Lambda}
\def \De {\Delta}
\def \DH {\Delta_{\rm HI}}
\newcommand{\de}{\delta}
\def \b {\beta}
\def \al {\alpha}
\def \ka {\kappa}
\def \Ga {\Gamma}
\def \ga {\gamma}
\def \si {\sigma}
\def \Si {\Sigma}
\def \ep {\epsilon}
\def \om {\omega}
\def \Om {\Omega}
\def \lap {\triangle}
\def \ep {\epsilon}


%%%%%%%%%%%%%%%%%%%%%%%%%%%%%%%%%%%
%Special definitions for this paper
%%%%%%%%%%%%%%%%%%%%%%%%%%%%%%%%%%%

\newcommand{\MyRed}{\color [rgb]{0.8,0,0}}
\newcommand{\MyGreen}{\color [rgb]{0,0.7,0}}
\newcommand{\MyBlue}{\color [rgb]{0,0,0.8}}
\newcommand{\MyBrown}{\color [rgb]{0.8,0.4,0.1}}
\newcommand{\MyPurple}{\color [rgb]{0.6,0.0,0.6}}
\def\GV#1{{\MyRed [GV: #1]}}
\def\RD#1{{\MyGreen [RD:  {\tt #1}]}} 
\def\RDt#1{{\MyGreen #1}}   
\def\GM#1{{\MyBlue [GM: #1]}}  
\def\GF#1{{\MyPurple [GF: #1]}}    



\newcommand{\ie}{\emph{i. e.}}
\newcommand{\cf}{\emph{cf.}}
\newcommand{\etal}{\emph{et al.}\xspace}
\newcommand{\eg}{\emph{e. g.}}

\newcommand{\Scal}{\mathcal S}
\newcommand{\DD}{\mathcal D}
\newcommand{\EE}{\mathcal E}
\newcommand{\MM}{\mathcal M}
\newcommand{\HH}{\mathcal H}

\newcommand{\Real}{\mathbb{R}}
\newcommand{\bn}{\boldsymbol{n}}
\newcommand{\bv}{\boldsymbol{v}}
\newcommand{\bx}{\boldsymbol{x}}
\newcommand{\bnabla}{\boldsymbol{\nabla}}
\newcommand{\bell}{\boldsymbol{\ell}}
\newcommand{\bal}{\boldsymbol{\alpha}}

%%%%%%%%%%%%%%%%%%%%%%%%%%%%%%%%%%%%%%%%%%



\title{N-body simulation of Kessence  }

\author[a]{.}
\author[a]{, .}
\author[a]{, .}
\author[a]{,.}
%\author[a]{, Francesco Montanari}


\affiliation[a]{
Universit\'e de Gen\`eve, D\'epartement de Physique Th\'eorique and CAP,
24 quai Ernest-Ansermet, CH-1211 Gen\`eve 4, Switzerland
}
%\affiliation[b]{Dipartimento di Fisica, Universit\`a di Bari, 
%Via G. Amendola 173, 70126 Bari, Italy}
%\affiliation[c]{Osservatorio Astronomico di Trieste, Universit\`a degli Studi di Trieste, Via Tiepolo 11, 34143 Trieste, Italy}

\emailAdd{farbod.hassani@unige.ch}
\emailAdd{martin.kunz@unige.ch}
\emailAdd{..}
\emailAdd{..}

\abstract{
}



%\keywords{
%Cosmology: 
%\vskip13pt plus8pt minus11pt
%\noindent{\bfseries\large\sffamily{Preprints:}} CERN-PH-TH-2015-132
%}

\begin{document}

\maketitle


%%%%%%%%%%%%%%%%%%%%%%%%%%%%%%%%%%%%%%%%%%%%%%%%%%%%%%%%%%%%%%%%%%%
\section{Introduction}
\label{Sec1}
\setcounter{equation}{0}
%%%%%%%%%%%%%%%%%%%%%%%%%%%%%%%%%%%%%%%%%%%%%
\section{Question:}
{\color{red} What is $\delta P/\delta\rho$ here and is it comparable with other papers?} \\
Discuss with Martin about the red term $P_{XX}$ why we do not agree? \\
What about "a" factor from physical to conformal perturbation in Stress tensor??  No we dont need it since from the begining everyhthing is in conformal time! \\
-Scale factror difference in $T_{0i}$

%%%%%%%%%%%%%%%%%%%%%%%%%%%%%%%%%%%%%%%%%%%%%
\section{Todo:}
-Is $X$ function of (t,x) or the field? since if it is the function of field derivative and field derivative is independent of field so $\partial X/\partial \varphi=0$. It is not allowed to take $X (t,\vec{x})$ when we know its functionality and it is not a function of $\varphi$. Why we cant write the equations 6,16-20? so the difference is   $- \mathcal{H} (1+w) $ in my calculation of $T_{00}$ instead of $-3 \mathcal{H}(1+w)$\\
  - A mistake in gauge transformation, where is it? \\
  - Are the equations in box are true?\\
  - The difference between "parabolic" and "elliptic" vector method? Since  I want to define "$T_i^0$" of k-essence... \\
  - What are the checks should be done? Gevolution transfer function at z=0 compared with hi-class results? The effect of k-essence field on matter power? Stability tests? what should be done exactly for stability tests? \\
   -vector elliptic, The difference between vector elliptic and parabolic?\\
  - $\mathcal{H}'$  in the code?! does $(\mathcal{H}^{(n+1)}-\mathcal{H}^{n})/d\tau$ makes sense? \\
 - Why in Gevolution source, $\Phi$ and $\chi$ has the same name in Fourier space? "scalarFT"? \\
 -According to the equation of 120, we need to have two mode $k$ and $k'$ in Fourier space to solve the field equation?! How we should solve it?!
  - Check stress tensor turning on vector elliptic, what is vector elliptic...? \\
  - Implement the IC for $\pi'$ in Gevolution. \\
  - The IC from hi-class, do some checks to find where it goes crazy! \\
  - Write down the full field update equation in theory and implement in the code and track the transfer function!\\
The updating metric/ particles in the sub steps of field update. \\
- Background results.
- Perturbation results, for $\phi$ and $T_{\mu \nu}$  \\
-Try to solve the differential equation in mathematica in 1D, for $c_s^2 ->0$. \\
-Check estimator method, what is the error? \\
- Do the calculation for  kessence $T_{\mu \nu}$, discuss about perturbation in conformal time and physical time. \\
- Compare the linear solution with hi-class results. \\
-Then do non linear run and compare \\
- Solve the initial condition problem. \\
- Put the result for initial condition which is produced by Gevolution. \\
-Fix the problem of kessence $T_{\mu \nu}$ \\
-Solve the differential equation for $\phi$ and see how it should behave. \\
- Add Lorenzo's file for looking at the field. \\
- compute $\delta_{\pi}$ and $\theta_{\pi}$ \\
Note that there is an error because in each loop we assume that $\Psi'$ and $\Phi'$ for kessence updates are constant!!! For 10 time update forexample!
Also according to leap frog and the fact that we update $\pi'$ half step in the first loop while we do not update Hubble constant! so we are making slightly different initial condition which we assume does not matter since the ODE goes to the attractor .. \\
Some points: The units are very important, like wavenumber which is $h/Mpc$ and $\pi$ is in unit of $Mpc$ and $'$ in Gevolution is in terms of conformal time which is in box units!. \\
%\label{Sec2}
%\setcounter{equation}{0}
%%%%%%%%%%%%%%%%%%%%%%%%%%%%%%%%%%%%%%%%%%%%%%%%

 \section{K-essence field equation from EFT action}
 
 We take the metric in ADM form as below,
  \be
  ds^2= -N (t,\vec{x}) ^2 d t^2+ h_{ij } (t,\vec{x}) \Big( dx^i+N^i (t,\vec{x}) dt   \Big) \Big( dx^j+N^j (t,\vec{x}) dt   \Big)
  \ee
  where 
  \be
  N(t,\vec{x})= \bar{N} (t) e^{\epsilon \delta N (t,\vec{x})}
  \ee
  \be
  N^i=\epsilon \sigma^{0i} (t,\vec{x})
  \ee
  \be
    h_{ij}=a^2 \Big( e^{2 \zeta (t,\vec{x}) \epsilon} \delta_{ij} + \epsilon \sigma_{ij} (t,\vec{x})   \Big)
  \ee
  $\epsilon$ shows the order of terms in the scheme.\\
  For the first order equations we can define $\sigma_{ij}=(\partial_i \partial_j- \frac{\nabla^2}{3} \delta_{ij}) B(t,\vec{x}) \epsilon$ and $N^i=\delta ^{ij} \partial_j \psi (t,\vec{x}) \epsilon$  since we can separate the scalar, vector and tensor equations. \\
  On the other hand in second order equations we do observe the mixing of the scalar, vector and tensor equations according to $ T^{\mu \nu} \frac{\delta g_{\mu \nu}}{\delta (scalars)}$, which cannot be written as a derivative of a scalar equation and suggest general definition of  $\sigma_{ij}$ with four degrees of freedom (1 scalar, 1 vector and 2 tensor degrees of freedom). Since here we are only interested in  scalar field equation we do not care about the details of $\sigma_{ij}$.
  \section{Definitions}
 The inverse of the metric is defined by the inverse of the  matrix.
\be
N_i=h_{ij} N^j
\ee
Christoffel symbols:
\be
\Gamma_{\zeta \rho}^{\mu}= \frac{g^{\mu \xi }}{2} \left(  g_{\xi \zeta ,\rho }+ g_{\xi \rho ,\zeta } - g_{\rho \zeta ,\xi }   \right )
\ee
\be
K_{ij}=\frac{1}{2 N (t,\vec{x})} \left [  \dot{h}_{ij} - \nabla_{i} N_{j} - \nabla_{j} N_{i}  \right ]= \frac{1}{2 N (t,\vec{x})} \left [  \dot{h}_{ij} - \partial_{i} N_{j} - \partial_{j} N_{i} -2  \Gamma_{i j}^{l} N_l  \right ] 
\ee
\be
\delta K= K_i^i(t,\vec{x}) -\bar {K}_i^i (t)
\ee
Full metric,
\be
g_{00}= -N^2(t,\vec{x})+ h_{ij} N^i N^j \,,\, g_{ij}=h_{ij} \, , \, g_{0i}=g_{i0}=h_{ij}N^j
\ee
Riemann tensor
\be
R^{\rho}_{\sigma \mu \zeta}= \partial_{\mu} \Gamma_{\zeta \sigma}^{\rho}- \partial_{\zeta} \Gamma_{\mu \sigma}^{\rho} + \Gamma_{\mu \lambda}^{\rho} \Gamma_{\zeta \sigma}^{\lambda} -  \Gamma_{\zeta \lambda}^{\rho} \Gamma_{\mu \sigma}^{\lambda}
\ee
Ricci tensor;
\be
R_{\mu \rho}=R ^{\eta}_{\mu  \eta  \rho}
\ee
Ricci scalar;
\be
R=g^{\mu \rho} R_{\mu \rho}
\ee
 
 \subsection{Stuckelberg trick}
\be
f(t) \longrightarrow f(t) +  \dot{f} (t) \pi+ \frac{1}{2} \ddot{f }(t) \pi^2  + \frac{1}{6} \dddot{f }(t) \pi^3
\ee
\be
\Lambda(t) \longrightarrow \Lambda(t) +  \dot{\Lambda} (t) \pi+ \frac{1}{2} \ddot{\Lambda}(t) \pi^2 + \frac{1}{6} \dddot{\Lambda }(t) \pi^3
\ee
\be
M_2^4(t) \longrightarrow M_2^4(t) +  \dot{ M_2^4} (t) \pi+ \frac{1}{2}    \ddot{ M_2^4 }(t)  \pi^2 + \frac{1}{6} \dddot{M_2^4 }(t) \pi^3
\ee
\be
m_3^3(t) \longrightarrow m_3^3(t) +  \dot{m_3^3} (t) \pi+ \frac{1}{2} \ddot{m_3^3}(t) \pi^2 + \frac{1}{6} \dddot{m_3^3 }(t) \pi^3
\ee
\be
g^{00} \longrightarrow g^{00} + 2 g^{0 \mu} \partial_{\mu} \pi + g^{\rho \nu} \partial_{\rho} \partial_{\nu} \pi
\ee
\be
\partial_0 \longrightarrow \left( 1- \dot{\pi} - \dot{\pi}^2\right) \partial_0
\ee
\be
\partial_i \longrightarrow \partial_i-  \left( 1- \dot{\pi} \right) \partial_i \pi \partial_0
\ee
\be
N \longrightarrow N \left(1-\dot{\pi} + \dot{\pi}^2+N^i\partial_i \pi + \frac{1}{2} N^2 h^{ij} \partial_i \pi \partial_j \pi \right)
\ee
\be
N^i \longrightarrow N^i(1- \dot{\pi} ) +(1- 2 \dot{\pi}) N^2 h^{ik} \partial_k \pi
\ee
\be
h_{ij} \longrightarrow h_{ij}- N_i \partial_j \pi -N_j \partial_i - N^2\partial_i \pi \partial_j \pi
\ee
%\begin{align}
%\delta K  \longrightarrow &   \delta K -3 \left ( \dot{H} \pi +\frac{1}{2} \ddot{H} \pi^2 \right ) - (1-\dot{\pi}) N h^{ij} \partial_i \partial_j \pi +\frac{1}{2} \partial_i h^{ij} \partial_j \pi  \nonumber \\ &+\frac{H}{2 a^2} \delta ^{ij}\partial_i \pi \partial_j \pi + \frac{2}{a^2} \delta ^{ij} \partial_i  \pi  \partial_j    \dot{\pi} -\frac{2}{a^2} \delta ^{ij} \partial_i N \partial_j \pi
%\end{align}
The EFT action is;
\be
S=\sqrt{-g} \left [ \frac{M_*^2}{2} f(t) R -\Lambda (t) -c(t) g^{00} +\frac{M_2^4(t)}{2} \left (g^{00} + \frac{1}{\bar{N}^2} \right )    -  \frac{m_3^3(t)}{2} \delta K  \left (g^{00} + \frac{1}{\bar{N}^2} \right )    \right ]
\ee

  Then the scalar field dynamics is obtained by varying the action with respect to the $\pi$. \\
  To change the gauge from unitary to Newtonian we use the following transformation in the variables. (Note that from now on we follow the notation of Gevolution where $\Psi$ is perturbation in time component and $\Phi$ is for spatial component while EFT papers are opposite)
 \be
 \delta N \rightarrow \Psi \, , \, \zeta \rightarrow-\Phi \, , \, \psi   \rightarrow0,  \,  \,  B \rightarrow0
 \ee

\begin{align}
\frac{1}{\sqrt{-g}} \frac{\delta S}{\delta \pi}|_{\text{First order} }&=  B_{\Psi} \Psi+  B_{\dot{\Psi}} \dot{\Psi} +
B_{\Phi} \Phi + B_{\dot{\Phi}} \dot{\Phi}  + B_{\ddot{\Phi}} \ddot{\Phi}+B_{\pi} \pi +   B_{\dot{\pi}} \dot{\pi} + B_{\ddot{\pi}} \ddot{\pi}  
\nonumber  \\& 
- \frac{k^2}{a^2} \left( B^{(2)}_{\Psi}\Psi +  B^{(2)}_{\Phi}\Phi+ B^{(2)}_{\dot{\Phi}} \dot{\Phi} + B^{(2)}_{\pi}\pi \right) + \frac{k^4}{a^4} \left(B^{(4)}_{\Phi}\Phi +B^{(4)}_{\pi}\pi  \right)
\end{align}
where,
\be
B_{\Psi}=12 c H+2 \dot{c} +3 m_3^3 (3H^2+2\dot{H}) -6 M_*^2\dot{f} (\dot{H} + 2H^2) + 3H \left[ 4M_2^4 +\dot{(m_3^3)}\right]+4 \dot{(M_2^4)}
\ee
The last equation is different with Essential building paper, because of a typo in the paper. Moreover in eq. 191 the second one is not equivalent to first one, there is a sign difference according to eq. 153 and taking derivative.
\be
B_{\dot{\Psi}}=2c + 4 M_2^4 +3 H (m_3^3 -M_*^2 \dot{f})
\ee
\be
B_{\Phi}=0
\ee
\be
B_{\dot{\Phi}}=3 \left[ 2c +3 H m_3^3-4 H M_*^2 \dot{f} +\dot{m_3^3}\right]
\ee
\be
B_{\ddot{\Phi}}=3( m_3^3 -M_*^2 \dot{f})
\ee
\be
B_{{\pi}}=- \Big[ -3\dot{m_3^3} \dot{H} - 6 \dot{H} c + 3 M_*^2 (\ddot{H} +4 H \dot{H})\dot{f} - 9 H \dot{H} m_3^3- 3 m_3^3 \ddot{H}  \Big]
\ee
\be
B_{\dot{\pi}}=- 2\Big[ 3 H (c+ 2 M_2^4) +\dot{c} + 2\dot{M_2^4} \Big]
\ee
\be
B_{\ddot{\pi}}=- 2\Big[  c+ 2 M_2^4 \Big]
\ee
\be
B^{(2)}_{{\Psi}}= \Big[  m_3^3  - M_*^2 \dot{f}\Big]
\ee
\be
B^{(2)}_{{\Phi}}= 2  M_*^2 \dot{f} 
\ee
\be
B^{(2)}_{\dot{\Phi}}=
  0
\ee
\be
B^{(2)}_{{\pi}}=
 \Big[ 2c  + \dot{m_3^3}+ H m_3^3 \Big]
\ee
\be
B^{(4)}_{{\Phi}}=0
\ee
\be
B^{(4)}_{{\pi}}=
0
\ee
The relevant second order, short wave correction terms in Fourier space are,
\begin{align}
 \frac{1}{\sqrt{-g}} \frac{\delta S}{\delta \pi}|_{\text{short wave} }&=  \int \int d^3k d^3 k' e^{i(\vec{k}+\vec{k}') . \vec{x}}  \Bigg [   -\frac{k^2}{a^2} C^{(2)}_{\Psi \Psi} \Psi \Psi  -\frac{k^2}{a^2} C^{(2)}_{\Psi \Phi} \Psi \Phi
   -\frac{k^2}{a^2} C^{(2)}_{\Psi \pi} \Psi \pi -\frac{k^2}{a^2} C^{(2)}_{\Phi \Phi} \Phi \Phi 
  \nonumber  \\& 
     -\frac{k^2}{a^2} C^{(2)}_{\Phi \pi} \Phi \pi 
   -\frac{k^2}{a^2} C^{(2)}_{\pi \Psi} \pi \Psi  -\frac{k^2}{a^2} C^{(2)}_{\pi \Phi} \pi \Phi   -\frac{k^2}{a^2} C^{(2)}_{\pi \pi} \pi \pi 
  -\frac{k^2}{a^2} C^{(2)}_{{\pi} \dot{\Phi}} {\pi} \dot{\Phi}  -\frac{k^2}{a^2} C^{(2)}_{{\pi} \dot{\Psi}} {\pi} \dot{\Psi}  
  %//////////////////////////
  \nonumber  \\& 
  -\frac{k^2}{a^2} C^{(2)}_{{\pi} \dot{\pi}} {\pi} \dot{\pi} 
  -\frac{k^2}{a^2} C^{(2)}_{\dot{\Phi} \pi} \dot{\Phi} \pi   -\frac{k^2}{a^2} C^{(2)}_{\dot{\Psi} \pi} \dot{\Psi} \pi 
  -\frac{k^2}{a^2} C^{(2)}_{\dot{\pi} \pi} \dot{\pi} \pi  -\frac{k^2}{a^2} C^{(2)}_{\dot{\pi} \Psi} \dot{\pi} \Psi 
      \nonumber \\&
      %///////////////////
  -\frac{\vec{k}.\vec{k}'}{a^2}  C^{1,1}_{\Phi \Phi} \Phi \Phi  -\frac{\vec{k}.\vec{k}'}{a^2}  C^{1,1}_{\Phi \Psi} \Phi \Psi 
 -\frac{\vec{k}.\vec{k}'}{a^2}  C^{1,1}_{\Phi \pi} \Phi \pi  -\frac{\vec{k}.\vec{k}'}{a^2}  C^{1,1}_{\Psi \pi} \Psi \pi   -\frac{\vec{k}.\vec{k}'}{a^2}  C^{1,1}_{\Psi \Psi} \Psi \Psi 
 \nonumber \\ &
 -\frac{\vec{k}.\vec{k}'}{a^2}  C^{1,1}_{\pi \pi} \pi \pi     -\frac{\vec{k}.\vec{k}'}{a^2}  C^{1,1}_{\dot{\Phi} \pi} \dot{\Phi} \pi -\frac{\vec{k}.\vec{k}'}{a^2}  C^{1,1}_{\dot{\pi} \pi} \dot{\pi} \pi
 +  C_{\ddot{\pi}} (\Psi,\Phi,\pi) \ddot{\pi} \Bigg ] 
  \text{ .}
\end{align}
Where non zero terms are,
\be
C^{(2)}_{\Phi \Phi}= 4 M_*^2 \dot{f} \text{ .}
\ee
\be
C^{(2)}_{\Phi \pi}= 2  M_*^2 \ddot{f}     \text{ .}
\ee
\be
C^{(2)}_{\Psi \Psi}=  - m_3^3 \text{ .}
\ee
\be
C^{(2)}_{\Psi \Phi}= 2(m_3^3 - M_*^2 \dot{f}) \text{ .}
\ee
\be
C^{(2)}_{\Psi \pi}=  (\dot{m_3^3} -M_*^2 \ddot{f}) \text{ .}
\ee
\be
C^{(2)}_{\Psi \dot{\pi}}= m_3^3\text{ .}
\ee
\be
C^{(2)}_{\pi \Phi}=4c + 2 H m_3^3 + 2 \dot{m_3^3}\text{ .}
\ee
\be
C^{(2)}_{\pi \Psi}= -4 M_2^4-4 H m_3^3\text{ .}
\ee
\be
C^{(2)}_{\pi \pi}=  -3m_3^3 \dot{H} +  H \dot{m_3^3} + 2 \dot{c} +  \ddot{m_3^3}   \text{ .}
\ee
\be
C^{(2)}_{\pi \dot{\Phi}}=  -4 m_3^3  \text{ .}
\ee
\be
C^{(2)}_{\pi \dot{\Psi}}=  -m_3^3  \text{ .}
\ee
\be
C^{(2)}_{\pi \dot{\pi}}=  4 M_2^4  \text{ .}
\ee
%/////////////////////////////////////////////////////////
\be
C^{1,1}_{\Phi \Phi}=	- M_*^2 \dot{f}	\text{ .}
\ee
\be
C^{1,1}_{\Psi \Phi}= - (m_3^3 -M_*^2 \dot{f}) \text{ .}
\ee
\be
C^{1,1}_{\Phi \pi}= - \left (  m_3^3 H + 2c + \dot{m_3^3}  \right )      \text{ .}
\ee
\be
C^{1,1}_{\Psi \pi}=  2 (-m_3^3 H + c- 2 M_2^4 +\dot{m_3^3 })    \text{ .}
\ee
\be
C^{1,1}_{\Psi \Psi}= - M_*^2 \dot{f}	\text{ .}
\ee
\be
C^{1,1}_{\pi \pi}= -\frac{1}{2} \left (   m_3^3 H^2 - 2 (\dot{c} +2  \dot{M_2^4} -2 m_3^3 \dot{H})+H (-4 M_2^4 +\dot{m_3^3} )  \right )	\text{ .}
\ee
\be
C^{1,1}_{\dot{\Phi} \pi}=  -4 m_3^3	\text{ .}
\ee
\be
C^{1,1}_{\dot{\pi} \pi}=   2 \left( 4 M_2^4 - H m_3^3- \dot{m_3^3}\right) 	\text{ .}
\ee
\be
C^{1,1}_{\dot{\pi} \Psi}=  2 m_3^3	\text{ .}
\ee
\be
C_{\ddot{\pi}} (\Psi,\Phi,\pi)= 	- m_3^3 \frac{k^2}{a^2} {\pi}\text{ .}
\ee
For the k-essence case we have (note below equation 85 in \url{https://arxiv.org/pdf/1411.3712.pdf})
\begin{align}
& \alpha_B= \alpha_H=\alpha_M=\alpha_T=0 \nonumber \\ &
\alpha_K=\frac{2\bar{X} P _X + 4 \bar{X}^2 P_{XX}}{M^2 H^2 }  \nonumber \\ &
c_s^2=\frac{-2 \dot{H}}{\alpha_K H^2 } - \frac{\rho + P}{\alpha_K M^2 H^2}
\end{align}
After translation between two different language according to table 1 of  \url{https://arxiv.org/pdf/1411.3712.pdf} and equation 24-25 of \url{https://arxiv.org/pdf/1210.0201.pdf} we get,
\begin{align}
 & m_4^2=\tilde{m}_4^2=\bar{\lambda}=0 \nonumber  \\ &
 f=1 \longrightarrow M^2=3M_{pl}^2,  \;  \;  m_3^3=0 , \;  \; \alpha_k=\frac{2c +4 M_2^4}{M^2 H^2}=  \frac{\Omega (1+w)}{c_s^2} \\ \nonumber &
 \Lambda= \frac{\bar{\rho} (1-w)}{2}, \; \; c=\frac{\bar{\rho} (1+w)}{2}, \; \; 4 M_2^4=\bar{\rho} (1+w) (\frac{1}{c_s^2}-1)
\end{align}
\subsection{Friedmann equations}
In this language the Friedman equations are,
\be
3{H}^2 M_{pl}^2= \rho_m + \rho_{scf}
\ee
Note that in Gevolution we have,
\be
\mathcal{H}^2=\frac{8 \pi G}{3} (\Omega_m a^{-3} +\Omega_{rad} a^{-4} +\Omega_{kess} a^{-3(1+w)} +\Omega_L a^{0} )
\ee
Where the critical density in here assumed 1, ie. $H_0^2=\mathcal{H}_0^2=\frac{8 \pi G}{3}$
and 
\begin{align}
&\frac{\ddot{a}}{a} = - \frac{1}{6M^2_{pl}} \left (\rho_{tot} +3 P_{tot}\right ) \\ \nonumber &
3 H^2 + 2\dot{H} = \frac{-1}{M^2_{pl}} \left( P_{m} + P_{scf} (X, \varphi) \right)
\end{align}
Equivalently it can be written,
\be
\dot{H}= \frac{-(2 \bar{X} P_{,X} + \rho_m+ P_m)}{6 M^2_{pl}}
\ee
where
\be
\bar{X}=\frac{1}{2} \; \; P_{,X} = \bar{\rho} (1+w)
\ee
and,
\be
\frac{\dot{\Omega}}{\Omega}= -3 H(1+w) -\frac{2\dot{H}}{H}
\ee
which is the same as equation 3.5 of \url{https://arxiv.org/pdf/1404.3713.pdf} 
The non-zero terms are:\\
{\color{red} Note that it is assumed that "w" is constant, $\bar{\rho}$ is density of k-essence field and continuity equation for k-essence field is $\dot{\bar{\rho}} +3 H \bar{\rho} (1+w)=0 $}.
\\Note that in order to compare with other papers, like eq. 113 of \url{https://arxiv.org/pdf/1411.3712.pdf} , since in all the terms we have $\bar{\rho}$ we can divide the equation to $3 M_{pl}^2 H^2$ and write everything in terms of $\Omega$ or $\alpha_i$. So of we compare the below result with eq.113 we have 3$M_{pl}^2$ factor a sign difference.
\be
B_{\Psi}=12 c H+2 \dot{c}  + 3H ( 4M_2^4)+4 \dot{(M_2^4)} =  \frac{\dot{\bar{\rho}} (1+w)}{c_s^2} + 3 {H} \bar{\rho} (1+w) \Big( \frac{1}{c_s^2}+1 \Big)=3 {H} \bar{\rho} (1+w) \Big( 1- \frac{w}{c_s^2} \Big )
\ee
\be
B_{\dot{\Psi}}=2c + 4 M_2^4 =  \frac{\bar{\rho} (1+w)}{c_s^2}
\ee
\be
B_{\dot{\Phi}}=6 c = 3 \bar{\rho} (1+w)
\ee
\be
B_{{\pi}}= 6 \dot{H} c= 3 \dot{H} \bar{\rho} (1+w)
\ee
\be
B_{\dot{\pi}}=- 2\Big[ 3 H (c+ 2 M_2^4) +\dot{c} + 2\dot{M_2^4} \Big] = \frac{3 H w (1+w) \bar{\rho} }{c_s^2}
\ee
\be
B_{\ddot{\pi}}=- 2\Big[  c+ 2 M_2^4 \Big]=- \frac{  \bar{\rho}(1+w) }{c_s^2}
\ee
\be
B^{(2)}_{{\pi}}=
2c  =\bar{\rho}(1+w) 
\ee
First order terms several times has checked, everything seems cosistent.

\be
C^{(2)}_{\pi \Phi}=4c =  2 \bar{\rho} (1+w) \text{ .}
\ee
\be
C^{(2)}_{\pi \Psi}= -4 M_2^4=-\bar{\rho} (1+w) (\frac{1}{c_s^2}-1)\text{ .}
\ee
\be
C^{(2)}_{\pi \pi}=   2 \dot{c}=-3 H  \bar{\rho} (1+w) ^2   \text{ .}
\ee
\be
C^{(2)}_{\pi \dot{\pi}}=  4 M_2^4=\bar{\rho} (1+w) (\frac{1}{c_s^2}-1)  \text{ .}
\ee
\be
C^{1,1}_{\Phi \pi}= -   2c =-   \bar{\rho} (1+w)      \text{ .}
\ee
\be
C^{1,1}_{\Psi \pi}=  2 c- 4 M_2^4 =\bar{\rho} (1+w) (2-\frac{1}{c_s^2})      \text{ .}
\ee
\be
C^{1,1}_{\pi \pi}= -\frac{1}{2} \left (  - 2 (\dot{c} +2  \dot{M_2^4} )+H (-4 M_2^4 )  \right )=-\frac{\bar{\rho} H (1+w)} {2 c_s^2} \Big(2+3w+c_s^2  \Big) 		\text{ .}
\ee
\be
C^{1,1}_{\dot{\pi} \pi}=   2 (4 M_2^4)=2\bar{\rho} (1+w) (\frac{1}{c_s^2}-1) 	\text{ .}
\ee
The leading order terms are checked several times with Mathematica notebook. \\
The final equation is:
\begin{align} 
 &3 {H} \bar{\rho} (1+w) \Big( 1- \frac{w}{c_s^2} \Big )
 \Psi + \frac{ \bar{\rho} (1+w)}{c_s^2} \dot{\Psi} + 3 \bar{\rho} (1+w) \dot{\Phi} +3 \dot{H} \bar{\rho} (1+w) \pi + \frac{3 {H} \bar{\rho} (1+w) w}{c_s^2} \dot{\pi} -\frac{  \bar{\rho} (1+w)}{c_s^2} \ddot{\pi} 
 \nonumber \\ 
 &
+ \bar{\rho} (1+w) \frac{\nabla^2 \pi}{a^2}
%////////////////  Second order temrs
  +2  \bar{\rho} (1+w) \Phi  \frac{\nabla^2 \pi }{a^2}   
  %//////////////// 
  -   \bar{\rho} (1+w) (\frac{1}{c_s^2}-1)  \Psi \frac{\nabla^2 \pi }{a^2}   
  %////////////////
  - 3 H \bar{\rho} (1+w)^2 \pi \frac{\nabla^2 \pi }{a^2}  
      \nonumber \\ &
      %////////////////
        +  \bar{\rho} (1+w) (\frac{1}{c_s^2}-1)    \dot{\pi } \frac{\nabla^2 {\pi }}{a^2}   
        %//////////////// 
             -\bar{\rho} (1+w)  \frac{\nabla  \Phi . \nabla \pi }{a^2} 
   %//////////////// 
        +\bar{\rho} (1+w) (2-\frac{1}{c_s^2}) \frac{\nabla  \Psi . \nabla \pi }{a^2}   
   %//////////////// 
   \nonumber \\ &
 -\frac{\bar{\rho} H (1+w)} {2 c_s^2} \Big(2+3w+c_s^2  \Big)\frac{\nabla  \pi . \nabla \pi } {a^2}
    %//////////////// 
    +2  \bar{\rho} (1+w) (\frac{1}{c_s^2}-1)\frac{\nabla  \pi . \nabla \dot{\pi} } {a^2}
     =0
  \end{align}
Simplifying the expression gives,
\begin{align} 
 & 3 {H}  \Big( 1- \frac{w}{c_s^2} \Big )
\Psi + \frac{ 1}{c_s^2} \dot{\Psi} + 3 \dot{\Phi} +3 \dot{H} \pi + \frac{3 {H} w}{c_s^2} \dot{\pi} -\frac{  1}{c_s^2} \ddot{\pi} + \frac{\nabla^2 \pi }{a^2}
   % Second order terms
     +2   \Phi  \frac{\nabla^2 \pi }{a^2}   
  %//////////////// 
  -   (\frac{1}{c_s^2}-1)  \Psi \frac{\nabla^2 \pi }{a^2}   
        \nonumber \\ &
  %////////////////
  - 3 H (1+w)\pi \frac{\nabla^2 \pi }{a^2}  
      %////////////////
        +   (\frac{1}{c_s^2}-1)    \dot{\pi } \frac{\nabla^2 {\pi }}{a^2}   
        %//////////////// 
             - \frac{\nabla  \Phi . \nabla \pi }{a^2} 
   %//////////////// 
        +(2-\frac{1}{c_s^2}) \frac{\nabla  \Psi . \nabla \pi }{a^2}   
   %//////////////// 
 -\frac{H} {2 c_s^2} \Big(2+3w+c_s^2  \Big)\frac{\nabla  \pi . \nabla \pi } {a^2}
    \nonumber \\ &
    %//////////////// 
    +2   (\frac{1}{c_s^2}-1)\frac{\nabla  \pi . \nabla \dot{\pi} } {a^2}     =0 \label{fineq}
%  -(\frac{1}{c_s^2}-1) \nabla^2 \Psi \pi+ 2 \nabla^2 \Phi \pi - 3 H (1+w) \pi \nabla^2 \pi  + (\frac{1}{c_s^2}-1) \pi \nabla^2 \dot{\pi}   \nonumber \\ &+ (2-\frac{1}{c_s^2})\nabla \Psi \nabla \pi - \nabla \Phi \nabla \pi -\frac{H} {2 c_s^2} \Big(2+3w+c_s^2  \Big) \nabla \pi \nabla \pi =0
  \end{align} 
  We have checked that the top equation agrees with equation 113 of \url{https://arxiv.org/pdf/1411.3712.pdf} up to first order. 
  \subsection{The equation in terms of conformal time}
  To express the last expression in terms of conformal time we only need to follow the following transformations, \\
  \be
  a d \tau= dt
  \ee
  The relation between Hubble and conformal Hubble is;
\be
\mathcal{H} (\tau)=\frac{1}{a(\tau) }\frac{d a(\tau)}{d \tau }= \frac{1}{a(t) } \frac{d a (t) }{d t} \frac{d t }{ d\tau}= a H(t)
\ee
and for the derivative,
\begin{align}
& \dot{H}= \frac{-\mathcal{H}^2+ \mathcal{H}'}{a^2} \nonumber \\ &
\mathcal{H}'=a^2 \Big[ H^2 + \dot{H}\Big ]
\end{align}
The time derivative of any function of time like $\pi(t)$ transforms as,
\be
\pi(x,t)=\pi(x,\tau)
\ee
But both are the perturbation in physical time hypersurfaces.
\be
\dot{\pi}=\frac{\partial \pi}{\partial \tau}  \frac{\partial \tau}{\partial t} = \frac{1}{a(\tau)} \pi '
\ee
\be
\ddot{\pi}=\frac{\partial \tau}  {\partial t} \frac{\partial }{\partial \tau} (\frac{1}{a}\pi ' ) = \frac{1}{a} (-\frac{a ' \pi'}{a^2}+ \frac{\pi''}{a})= \frac{1}{a^2} \Big(- \mathcal{H} \pi' + \pi'' \Big)
\ee
So the equation \ref{fineq} becomes,
\begin{align} 
 & 3 \frac{\mathcal{H}
}{a} \Big( 1- \frac{w}{c_s^2} \Big )\Psi + \frac{ 1}{c_s^2} \frac{\Psi'}{a}+ 3 \frac{\Phi'}{a} + 3  \frac{-\mathcal{H}^2 + \mathcal{H}'}{a^2} \pi + \frac{3 \mathcal{H} w}{ a c_s^2} \frac{\pi'}{a}  -\frac{  1}{ a^2 c_s^2} \Big(- \mathcal{H} \pi' + \pi'' \Big) + \frac{\nabla^2 \pi }{a^2}
% Second order terms
\nonumber \\ &
     +2   \Phi  \frac{\nabla^2 \pi }{a^2}   
  %//////////////// 
  -   (\frac{1}{c_s^2}-1)  \Psi \frac{\nabla^2 \pi }{a^2}   
  %////////////////
  - 3 \mathcal{H} (1+w)\pi \frac{\nabla^2 \pi }{a^3}  
      %////////////////
        +   (\frac{1}{c_s^2}-1) \frac{ \pi' \nabla^2 {\pi }}{a^3}   
        %//////////////// 
             - \frac{\nabla  \Phi . \nabla \pi }{a^2} 
   %//////////////// 
        +(2-\frac{1}{c_s^2}) \frac{\nabla  \Psi . \nabla \pi }{a^2}   
   %//////////////// 
    \nonumber \\ &
 -\frac{\mathcal{H}} {2 a c_s^2} \Big(2+3w+c_s^2  \Big)\frac{\nabla  \pi . \nabla \pi } {a^2}
    %//////////////// 
    +2   (\frac{1}{c_s^2}-1)\frac{\nabla  \pi . \nabla {\pi'} } {a^3}       =0 \label{fineq}
%  -(\frac{1}{c_s^2}-1) \nabla^2 \Psi \pi+ 2 \nabla^2 \Phi \pi - 3 H (1+w) \pi \nabla^2 \pi  + (\frac{1}{c_s^2}-1) \pi \nabla^2 \dot{\pi}   \nonumber \\ &+ (2-\frac{1}{c_s^2})\nabla \Psi \nabla \pi - \nabla \Phi \nabla \pi -\frac{H} {2 c_s^2} \Big(2+3w+c_s^2  \Big) \nabla \pi \nabla \pi =0
  \end{align} 
  Multiplying to $-a^2 c_s^2$ gives:
%  \begin{empheq}[box=\tcbhighmath]{equation}
 \begin{align} 
 &\pi_{\text{phys}}'' - \mathcal{H} \Big (1+ 3w \Big)\pi_{\text{phys}}' -3 {a c_s^2 \mathcal{H}}\Big( 1- \frac{w}{c_s^2} \Big )\Psi -a \, {\Psi'}- 3 c_s^2 a \,{\Phi'} 
  -3  c_s^2 \Big({-\mathcal{H}^2 + \mathcal{H}'} \Big) \pi_{\text{phys}} 
 - c_s^2 {\nabla^2 \pi_{\text{phys}} }
             \nonumber
   \\
    &
    % Second order terms
     -2 c_s^2  \Phi  {\nabla^2 \pi_{\text{phys}} }  
  %//////////////// 
  +   (1-c_s^2)  \Psi {\nabla^2 \pi_{\text{phys}} }
  %////////////////
  +3 c_s^2 \mathcal{H} (1+w)\pi_{\text{phys}} \frac{\nabla^2 \pi_{\text{phys}} }{a}  
      %////////////////
        -   (1-c_s^2) \frac{ \pi_{\text{phys}}' \nabla^2 {\pi_{\text{phys}} }}{a}   
        %//////////////// 
             +c_s^2 {\nabla  \Phi . \nabla \pi_{\text{phys}} }
               \nonumber 
               \\
                &
   %//////////////// 
        -(2 c_s^2-1) {\nabla  \Psi . \nabla \pi_{\text{phys}} }  
   %//////////////// 
 +\frac{\mathcal{H}} {2 a} \Big(2+3w+c_s^2  \Big){\nabla  \pi_{\text{phys}} . \nabla \pi_{\text{phys}} } 
    %//////////////// 
    -2   (1-c_s^2)\frac{\nabla  \pi_{\text{phys}} . \nabla {\pi_{\text{phys}}'} } {a}       =0
   %  -(\frac{1}{c_s^2}-1) \nabla^2 \Psi \pi+ 2 \nabla^2 \Phi \pi - 3 H (1+w) \pi \nabla^2 \pi  + (\frac{1}{c_s^2}-1) \pi \nabla^2 \dot{\pi}   \nonumber \\ &+ (2-\frac{1}{c_s^2})\nabla \Psi \nabla \pi - \nabla \Phi \nabla \pi -\frac{H} {2 c_s^2} \Big(2+3w+c_s^2  \Big) \nabla \pi \nabla \pi =0
  \end{align} 
%\end{empheq}
It is very important to note that $\pi_{\text{phys}}$ here is the perturbation on physical time hypersurfaces, while in hi-class (equation 2.16 of https://arxiv.org/pdf/1605.06102.pdf) it is perturbation on conformal time hypersurfaces. The relation between $\pi_{\text{physical}}$ and $\pi_{\text{conformal}}$ is as following,
\be
\pi_{\text{conf}}= \frac{\delta \varphi_{\text{phys}}}{\bar{\dot{\varphi}} \, a} = \frac{\pi_{\text{phys}}}{a}
\ee
{\color{blue}
\be
\pi_{phys}'=(a \pi_{con})'=a (\mathcal{H} \pi_{con}+ \pi'_{con})
\ee
}
{\color{blue}
\be
\pi_{phys}''=(a \pi_{con})''=a( \mathcal{H}' \pi_{con}+2 \mathcal{H} \pi'_{con} +\pi''_{con}+ \mathcal{H}^2 \pi_{con} )
\ee
}
Substituting gives,
\begin{align} 
 &\pi_{\text{phys}}'' - \mathcal{H} \Big (1+ 3w \Big)\pi_{\text{phys}}' -3 {a c_s^2 \mathcal{H}}\Big( 1- \frac{w}{c_s^2} \Big )\Psi -a \, {\Psi'}- 3 c_s^2 a \,{\Phi'} 
  -3  c_s^2 \Big({-\mathcal{H}^2 + \mathcal{H}'} \Big) \pi_{\text{phys}} 
 - c_s^2 {\nabla^2 \pi_{\text{phys}} }
             \nonumber
   \\
    &
    % Second order terms
     -2 c_s^2  \Phi  {\nabla^2 \pi_{\text{phys}} }  
  %//////////////// 
  +   (1-c_s^2)  \Psi {\nabla^2 \pi_{\text{phys}} }
  %////////////////
  +3 c_s^2 \mathcal{H} (1+w)\pi_{\text{phys}} \frac{\nabla^2 \pi_{\text{phys}} }{a}  
      %////////////////
        -   (1-c_s^2) \frac{ \pi_{\text{phys}}' \nabla^2 {\pi_{\text{phys}} }}{a}   
        %//////////////// 
             +c_s^2 {\nabla  \Phi . \nabla \pi_{\text{phys}} }
               \nonumber 
               \\
                &
   %//////////////// 
        -(2 c_s^2-1) {\nabla  \Psi . \nabla \pi_{\text{phys}} }  
   %//////////////// 
 +\frac{\mathcal{H}} {2 a} \Big(2+3w+c_s^2  \Big){\nabla  \pi_{\text{phys}} . \nabla \pi_{\text{phys}} } 
    %//////////////// 
    -2   (1-c_s^2)\frac{\nabla  \pi_{\text{phys}} . \nabla {\pi_{\text{phys}}'} } {a}       =0
   %  -(\frac{1}{c_s^2}-1) \nabla^2 \Psi \pi+ 2 \nabla^2 \Phi \pi - 3 H (1+w) \pi \nabla^2 \pi  + (\frac{1}{c_s^2}-1) \pi \nabla^2 \dot{\pi}   \nonumber \\ &+ (2-\frac{1}{c_s^2})\nabla \Psi \nabla \pi - \nabla \Phi \nabla \pi -\frac{H} {2 c_s^2} \Big(2+3w+c_s^2  \Big) \nabla \pi \nabla \pi =0
  \end{align} 
%  \begin{empheq}[box=\tcbhighmath]{equation}
 \begin{align} 
 &{\color{blue} \mathcal{H}' \pi_{con}+2 \mathcal{H} \pi'_{con} +\pi''_{con}+ \mathcal{H}^2 \pi_{con} }- \mathcal{H} \Big (1+ 3w \Big)({\color{blue} \mathcal{H} \pi_{con}+ \pi'_{con}}) -3 { c_s^2 \mathcal{H}}\Big( 1- \frac{w}{c_s^2} \Big )\Psi 
    - \, {\Psi'}
 - 3 c_s^2  \,{\Phi'} 
             \nonumber
   \\
    &
  -3  c_s^2 \Big({-\mathcal{H}^2 + \mathcal{H}'} \Big) \pi_{\text{conf}} 
 - c_s^2 {\nabla^2 \pi_{\text{conf}} }
    % Second order terms
     -2 c_s^2  \Phi  {\nabla^2 \pi_{\text{conf}} }  
  %//////////////// 
  +   (1-c_s^2)  \Psi {\nabla^2 \pi_{\text{conf}} }
  %////////////////
  +3 c_s^2 \mathcal{H} (1+w)\pi_{\text{conf}} {\nabla^2 \pi_{\text{conf}} }
              \nonumber
   \\
    &
      %////////////////
        -   (1-c_s^2) { \color{blue}(\mathcal{H} \pi_{con}+ \pi'_{con})} \nabla^2 {\pi_{\text{conf}} } 
        %//////////////// 
             +c_s^2 {\nabla  \Phi . \nabla \pi_{\text{conf}} }
   %//////////////// 
        -(2 c_s^2-1) {\nabla  \Psi . \nabla \pi_{\text{conf}} }  
   %//////////////// 
 +\frac{\mathcal{H}} {2 } \Big(2+3w+c_s^2  \Big){\nabla  \pi_{\text{conf}} . \nabla \pi_{\text{conf}} } 
                                 \nonumber
   \\
    &
    %//////////////// 
     -2   (1-c_s^2){\nabla  \pi_{\text{conf}} . \nabla {\color{blue} {(\mathcal{H} \pi_{con}+ \pi'_{con})} }}       =0
   %  -(\frac{1}{c_s^2}-1) \nabla^2 \Psi \pi+ 2 \nabla^2 \Phi \pi - 3 H (1+w) \pi \nabla^2 \pi  + (\frac{1}{c_s^2}-1) \pi \nabla^2 \dot{\pi}   \nonumber \\ &+ (2-\frac{1}{c_s^2})\nabla \Psi \nabla \pi - \nabla \Phi \nabla \pi -\frac{H} {2 c_s^2} \Big(2+3w+c_s^2  \Big) \nabla \pi \nabla \pi =0
  \end{align} 
%\end{empheq}

%\begin{empheq}[box=\tcbhighmath]{equation}
% \begin{align} 
% &{\color{blue}\mathcal{H}' \pi_{con}+\mathcal{H} \pi'_{con} +\pi''_{con} }- \mathcal{H} \Big (1+ 3w \Big)({\color{blue} \mathcal{H} \pi_{con}+ \pi'_{con}}) -3 { c_s^2 \mathcal{H}}\Big( 1- \frac{w}{c_s^2} \Big )\Psi - \, {\Psi'}- 3 c_s^2  \,{\Phi'} 
%           \nonumber
%   \\
%    &
%  -3  c_s^2 \Big({-\mathcal{H}^2 + \mathcal{H}'} \Big) \pi_{\text{conf}} 
% - c_s^2 {\nabla^2 \pi_{\text{conf}} }
%    % Second order terms
%     -2 c_s^2  \Phi  {\nabla^2 \pi_{\text{conf}} }  
%  %//////////////// 
%  +   (1-c_s^2)  \Psi {\nabla^2 \pi_{\text{conf}} }
%  %////////////////
%  +3 c_s^2 \mathcal{H} (1+w)\pi_{\text{conf}} {\nabla^2 \pi_{\text{conf}} }
%      %////////////////
%                                      \nonumber
%   \\
%    &
%        -   (1-c_s^2) { \pi_{\text{conf}}' \nabla^2 {\pi_{\text{conf}} }} 
%        %//////////////// 
%             +c_s^2 {\nabla  \Phi . \nabla \pi_{\text{conf}} }
%   %//////////////// 
%        -(2 c_s^2-1) {\nabla  \Psi . \nabla \pi_{\text{conf}} }  
%   %//////////////// 
% +\frac{\mathcal{H}} {2 } \Big(2+3w+c_s^2  \Big){\nabla  \pi_{\text{conf}} . \nabla \pi_{\text{conf}} } 
%                                 \nonumber
%   \\
%    &
%    %//////////////// 
%     -2   (1-c_s^2){\nabla  \pi_{\text{conf}} . \nabla {\pi_{\text{conf}}'} }       =0
%   %  -(\frac{1}{c_s^2}-1) \nabla^2 \Psi \pi+ 2 \nabla^2 \Phi \pi - 3 H (1+w) \pi \nabla^2 \pi  + (\frac{1}{c_s^2}-1) \pi \nabla^2 \dot{\pi}   \nonumber \\ &+ (2-\frac{1}{c_s^2})\nabla \Psi \nabla \pi - \nabla \Phi \nabla \pi -\frac{H} {2 c_s^2} \Big(2+3w+c_s^2  \Big) \nabla \pi \nabla \pi =0
%  \end{align} 
%\end{empheq}
%\begin{empheq}[box=\tcbhighmath]{equation}
 \begin{align} 
 &{\color{blue}\pi''_{conf} +\mathcal{H}(1- 3w) \pi'_{conf} } -3 { c_s^2 \mathcal{H}}\Big( 1- \frac{w}{c_s^2} \Big )\Psi - \, {\Psi'}- 3 c_s^2  \,{\Phi'} + {\color{blue}
 \Big( 3\mathcal{H}^2 (c_s^2 -w) + \mathcal{H}' (1-3c_s^2)\Big) \pi_{\text{conf}} }
           \nonumber
   \\
    &
 - c_s^2 {\nabla^2 \pi_{\text{conf}} }
    % Second order terms
     -2 c_s^2  \Phi  {\nabla^2 \pi_{\text{conf}} }  
  %//////////////// 
  +   (1-c_s^2)  \Psi {\nabla^2 \pi_{\text{conf}} }
  %////////////////
  +3 c_s^2 \mathcal{H} (1+w)\pi_{\text{conf}} {\nabla^2 \pi_{\text{conf}} }
      %////////////////
                                      \nonumber
   \\
    &
        -   (1-c_s^2)  { \color{blue}(\mathcal{H} \pi_{con}+ \pi'_{con}) } \nabla^2 {\pi_{\text{conf}} }
        %//////////////// 
             +c_s^2 {\nabla  \Phi . \nabla \pi_{\text{conf}} }
   %//////////////// 
        -(2 c_s^2-1) {\nabla  \Psi . \nabla \pi_{\text{conf}} }  
   %//////////////// 
                                    \nonumber
   \\
    &
 +\frac{\mathcal{H}} {2 } \Big(2+3w+c_s^2  \Big){\nabla  \pi_{\text{conf}} . \nabla \pi_{\text{conf}} } 
    %//////////////// 
     -2   (1-c_s^2){\nabla  \pi_{\text{conf}} . {\color{blue}  \nabla {  (\mathcal{H} \pi_{con}+ \pi'_{con})   }}}     =0
   %  -(\frac{1}{c_s^2}-1) \nabla^2 \Psi \pi+ 2 \nabla^2 \Phi \pi - 3 H (1+w) \pi \nabla^2 \pi  + (\frac{1}{c_s^2}-1) \pi \nabla^2 \dot{\pi}   \nonumber \\ &+ (2-\frac{1}{c_s^2})\nabla \Psi \nabla \pi - \nabla \Phi \nabla \pi -\frac{H} {2 c_s^2} \Big(2+3w+c_s^2  \Big) \nabla \pi \nabla \pi =0
  \end{align} 
%\end{empheq}
\begin{empheq}[box=\tcbhighmath]{equation}
 \begin{align} 
 &{ \pi''_{con} +\mathcal{H}(1- 3w) \pi'_{con} } +3 {  \mathcal{H}}\Big( -c_s^2+ {w} \Big )\Psi - \, {\Psi'}- 3 c_s^2  \,{\Phi'} + {
 \Big( 3\mathcal{H}^2 (c_s^2 -w) + \mathcal{H}' (1-3c_s^2)\Big) \pi_{\text{conf}} }
           \nonumber
   \\
    &
 - c_s^2 {\nabla^2 \pi_{\text{conf}} }
    % Second order terms
     -2 c_s^2  \Phi  {\nabla^2 \pi_{\text{conf}} }  
  %//////////////// 
  +   (1-c_s^2)  \Psi {\nabla^2 \pi_{\text{conf}} }
  %////////////////
  +3 c_s^2 \mathcal{H} (1+w)\pi_{\text{conf}} {\nabla^2 \pi_{\text{conf}} }
      %////////////////
                                      \nonumber
   \\
    &
        -   (1-c_s^2)  { (\mathcal{H} \pi_{con}+ \pi'_{con}) } \nabla^2 {\pi_{\text{conf}} }
        %//////////////// 
             +c_s^2 {\nabla  \Phi . \nabla \pi_{\text{conf}} }
   %//////////////// 
        -(2 c_s^2-1) {\nabla  \Psi . \nabla \pi_{\text{conf}} }  
   %//////////////// 
                                    \nonumber
   \\
    &
 +\frac{\mathcal{H}} {2 } \Big(2+3w+c_s^2  \Big){\nabla  \pi_{\text{conf}} . \nabla \pi_{\text{conf}} } 
    %//////////////// 
     -2   (1-c_s^2){\nabla  \pi_{\text{conf}} . {  \nabla {  (\mathcal{H} \pi_{con}+ \pi'_{con})   }}}     =0
   %  -(\frac{1}{c_s^2}-1) \nabla^2 \Psi \pi+ 2 \nabla^2 \Phi \pi - 3 H (1+w) \pi \nabla^2 \pi  + (\frac{1}{c_s^2}-1) \pi \nabla^2 \dot{\pi}   \nonumber \\ &+ (2-\frac{1}{c_s^2})\nabla \Psi \nabla \pi - \nabla \Phi \nabla \pi -\frac{H} {2 c_s^2} \Big(2+3w+c_s^2  \Big) \nabla \pi \nabla \pi =0
  \end{align} 
\end{empheq}
If we drop the "conf" from the fields but remembering that the fields are perturbation in constant conformal time hypersurfaces, we have,
\begin{empheq}[box=\tcbhighmath]{equation}
 \begin{align} 
 &{ \pi''+\mathcal{H}(1- 3w) \pi' } +3 {  \mathcal{H}}\Big( -c_s^2+ {w} \Big )\Psi - \, {\Psi'}- 3 c_s^2  \,{\Phi'} + {
 \Big( 3\mathcal{H}^2 (c_s^2 -w) + \mathcal{H}' (1-3c_s^2)\Big) \pi }
           \nonumber
   \\
    &
 - c_s^2 {\nabla^2 \pi }
    % Second order terms
     -2 c_s^2  \Phi  {\nabla^2 \pi }  
  %//////////////// 
  +   (1-c_s^2)  \Psi {\nabla^2 \pi}
  %////////////////
  +3 c_s^2 \mathcal{H} (1+w)\pi {\nabla^2 \pi }
      %////////////////
        -   (1-c_s^2)  { (\mathcal{H} \pi+ \pi') } \nabla^2 {\pi }
                                       \nonumber
   \\
    &
        %//////////////// 
             +c_s^2 {\nabla  \Phi . \nabla \pi}
   %//////////////// 
        -(2 c_s^2-1) {\nabla  \Psi . \nabla \pi }  
   %//////////////// 
 +\frac{\mathcal{H}} {2 } \Big(2+3w+c_s^2  \Big){\nabla  \pi . \nabla \pi} 
    %//////////////// 
     -2   (1-c_s^2){\nabla  \pi . {  \nabla {  (\mathcal{H} \pi+ \pi')   }}}     =0
  \end{align} 
\end{empheq}

The linear equation which is comparable with class is as following;
%\end{empheq}
\begin{empheq}[box=\tcbhighmath]{equation}
 \begin{align} 
 &{ \pi''_{con} +\mathcal{H}(1- 3w) \pi'_{con} } +3 {  \mathcal{H}}\Big( -c_s^2+ {w} \Big )\Psi - \, {\Psi'}- 3 c_s^2  \,{\Phi'} + {
 \Big( 3\mathcal{H}^2 (c_s^2 -w) + \mathcal{H}' (1-3c_s^2)\Big) \pi_{\text{conf}} }
           \nonumber
   \\
    &
 - c_s^2 {\nabla^2 \pi_{\text{conf}} } =0
    % Second order terms==0
  \end{align} 
\end{empheq}
%%%%%%%%%%%%%%%%%%%%%
The followed expression is needed to be the same as EFT papers equation with using the fact that they write the equation for perturbation on constant physical time hypersurfaces. \\
In order to compare the result with other results (ex. equation 113 of https://arxiv.org/pdf/1411.3712.pdf) we write down the coefficients in the paper and try to compare it with our result: \\
The coefficient of $\ddot{\pi}$ in the paper is:
\be
H^2 \alpha_K  \ddot{\pi}_{phys}= \frac{ H^2 \Omega (1+w)}{c_s^2} \ddot{\pi}_{phys}
\ee
by multiplying to $\frac{ c_s^2}{ a (1+w) \Omega H^2}$ we get 1! We divided by "a" since it is introduced by $\pi_{\text{conf}}$.  Moreover it is important to note that $\ddot{\pi}= \frac{-\mathcal{H} \pi' + \pi''}{a^2}$, so in our expression we also multiply to $a^2$ to get 1 for coefficient of $\pi''$ and $-\mathcal{H} \pi'$. So at the end we multiply all terms by $\frac{a c_s^2}{ (1+w) \Omega H^2}$. In sum,
\be
\frac{ H^2 \Omega (1+w)}{c_s^2} \ddot{\pi}_{phys} \times \frac{ a c_s^2}{  (1+w) \Omega H^2}  =  \pi''_{\text{conf}} - \mathcal{H} \pi'_{\text{conf}}
\ee
The next term is $\dot{\pi}$,
\be
 H^2 \alpha_k (3H +2 \frac{\dot{H}}{H} + \frac{\dot{\alpha_k}}{\alpha_k} ) \dot{\pi}_{phys} = - \frac{3 H^3 w (1+w) \Omega  }{c_s^2} \dot{\pi}_{phys} \xrightarrow{  \times \frac{ ac_s^2}{(1+w) \Omega H^2}} -3 w \mathcal{H} \pi'_{\text{conf}}
\ee
Where we have used $\frac{\dot{\alpha_k}}{\alpha_k} = \frac{\dot{\Omega}}{\Omega}= -3 H (1+w) -2 \frac{\dot{H}}{H}$ and $\dot{H}+ \frac{\rho_m + P_m}{2 M^2} = - \frac{2 \bar{X} P_{,X}}{2 M^2}=\frac{- \bar{\rho}(1+w)}{2 M^2} $.  \\
%Again by multiplying to $  \frac{ ac_s^2}{(1+w) \Omega H^2}$ we get 
%$-\frac{3 w H}{a}$ which is the coefficient of $\dot{\pi}_{conf}$ for us it would be $-{3 \mathcal{H} ^2w}$ which is coefficient of ${\pi}'_{conf}$ 
$\pi$ coefficient:
\be
6 \dot{H}(\dot{H} + \frac{\rho_m + P_m}{2M^2}) \pi_{phys} = -3 a^2 c_s^2 \dot{H} \pi_{\text{conf}} = -3 c_s^2(-\mathcal{H}^2 + \mathcal{H}') \pi_{\text{conf}}
\ee
Note that $\Omega= \frac{\rho}{3 M_{pl}^2 H^2}=  \frac{\rho}{M^2 H^2}$. \\
The coefficient of $\nabla^2 \pi$ is :
\be
2 (\dot{H} + \frac{\rho_m + P_m}{2M^2}) \frac{\nabla^2 \pi_{phys}}{a^2} \xrightarrow{  \times \frac{ ac_s^2}{(1+w) \Omega H^2}} -c_s^2  \nabla^2 \pi_{\text{conf}}
\ee
The coefficient of $\dot{\Psi}$, (note that $\Psi$ and $\Phi$ are not the same in our convention and the paper!)
\be
- H^2 \alpha_k \dot{\Psi} = -\frac{\Omega (1+w) }{c_s^2} \Psi'/a \xrightarrow{  \times \frac{ ac_s^2}{(1+w) \Omega H^2}} -\Psi'
\ee
The coefficient of $\dot{\Phi}$, ($\dot{\Psi}$ in the paper)
\be
6(\dot{H} + \frac{\rho_m + P_m}{2M^2}) \dot{\Phi} = \frac{ -3 \bar{\rho}(1+w)}{M^2}  {\Phi'}/a\xrightarrow{  \times \frac{ ac_s^2}{(1+w) \Omega H^2}}  -3 \, c_s^2 \Phi'
\ee
The coefficient of ${\Psi}$, (${\Phi}$ in the paper)
\begin{align}
 & \left [ 6(\dot{H} + \frac{\rho_m + P_m}{2M^2}) + H \alpha_K \left ( -3  H -2 \frac{\dot{H}}{H} - \frac{\dot{\alpha}_K}{\alpha_K } \right ) \right ] H \Psi= \Big( -3 H^2 \Omega (1+w) + \frac{\Omega (1+w)}{c_s^2} 3 H^2 w)\Big ) \mathcal{H}/a  \Psi \nonumber. \\ &
 \xrightarrow{  \times \frac{ ac_s^2}{(1+w) \Omega H^2}} -3 \mathcal{H} (c_s^2 -w )\Psi
\end{align}
So we get exactly the same first order equations as the references! \\
\subsection{Numeric olver}
Note that ${H'}$ can be determined by all the matter contents of the universe not by k-essence alone, the continuity equation for k-essence or matter gives the dynamics of density. \\
The field equation is:
we take $d \tau=\tau_{n+1}-\tau_n $ and $x_{i,j,k} $ as lattice point. We solve the differential equation numerically as following;
\be
\pi_v= {\pi}'
\ee
\be
\pi^{n}= \pi ^{n-1}+\pi_v ^{n-\frac{1}{2}} d \tau
\ee
\be \label{eq3}
\pi_v ^{n+\frac{1}{2}}=\pi_v ^{n-\frac{1}{2}} + {\pi''} ^{(n)}  d \tau
\ee

We define the laplacian in code as following,
\begin{align}
& \nabla^2 \pi =-\frac{\pi^{n}_{i-1,j,k}+\pi^{n}_{i+1,j,k} +\pi^{n}_{i,j-1,k} +\pi^{n}_{i,j+1,k}+\pi^{n}_{i,j,k-1}+\pi^{n}_{i,j,k+1} -6 \pi^{n}_{i,j,k}  }{ a^2 dx^2}  
\end{align}
Moreover in order to get scalar in the vertices the derivatives like $\nabla \pi . \nabla \pi $ should be defined symmetric .
So we can rewrite the equation \ref{eq3} as below;
\begin{align} 
 &{\color{blue}\pi''_{con} +\mathcal{H}(1- 3w) \pi'_{con} } -3 { c_s^2 \mathcal{H}}\Big( 1- \frac{w}{c_s^2} \Big )\Psi - \, {\Psi'}- 3 c_s^2  \,{\Phi'} + {\color{blue}
 \Big( 3\mathcal{H}^2 (c_s^2 -w) + \mathcal{H}' (1-3c_s^2)\Big) \pi_{\text{conf}} }
           \nonumber
   \\
    &
 - c_s^2 {\nabla^2 \pi_{\text{conf}} }
    % Second order terms
     -2 c_s^2  \Phi  {\nabla^2 \pi_{\text{conf}} }  
  %//////////////// 
  +   (1-c_s^2)  \Psi {\nabla^2 \pi_{\text{conf}} }
  %////////////////
  +3 c_s^2 \mathcal{H} (1+w)\pi_{\text{conf}} {\nabla^2 \pi_{\text{conf}} }
      %////////////////
                                      \nonumber
   \\
    &
        -   (1-c_s^2)  { \color{blue}(\mathcal{H} \pi_{con}+ \pi'_{con}) } \nabla^2 {\pi_{\text{conf}} }
        %//////////////// 
             +c_s^2 {\nabla  \Phi . \nabla \pi_{\text{conf}} }
   %//////////////// 
        -(2 c_s^2-1) {\nabla  \Psi . \nabla \pi_{\text{conf}} }  
   %//////////////// 
                                    \nonumber
   \\
    &
 +\frac{\mathcal{H}} {2 } \Big(2+3w+c_s^2  \Big){\nabla  \pi_{\text{conf}} . \nabla \pi_{\text{conf}} } 
    %//////////////// 
     -2   (1-c_s^2){\nabla  \pi_{\text{conf}} . {\color{blue}  \nabla {  (\mathcal{H} \pi_{con}+ \pi'_{con})   }}}     =0
   %  -(\frac{1}{c_s^2}-1) \nabla^2 \Psi \pi+ 2 \nabla^2 \Phi \pi - 3 H (1+w) \pi \nabla^2 \pi  + (\frac{1}{c_s^2}-1) \pi \nabla^2 \dot{\pi}   \nonumber \\ &+ (2-\frac{1}{c_s^2})\nabla \Psi \nabla \pi - \nabla \Phi \nabla \pi -\frac{H} {2 c_s^2} \Big(2+3w+c_s^2  \Big) \nabla \pi \nabla \pi =0
  \end{align} 
\begin{align} 
 &\pi_v ^{n+\frac{1}{2}}=\pi_v ^{n-\frac{1}{2}} - d \tau \Big [ \mathcal{H}^{(n)} (1-3w)\frac{(\pi_{v  \; {i,j,k}}^{n+\frac{1}{2}} +\pi_{v \; {i,j,k}}^{n-\frac{1}{2}} )}{2} -3 {c_s^2 \mathcal{H}^{(n)}}\Big( 1- \frac{w}{c_s^2} \Big )\Psi^{(n) }
 -  \frac{{\Psi}^{(n)}-{\Psi}^{(n-1)} }{d \tau}
    \nonumber
     \\
      &
      - 3c_s^2  \, \frac{{\Phi}^{(n)}-{\Phi}^{(n-1)} }{d \tau}    
   +\Big( 3\mathcal{H}^2 (c_s^2 -w) + \mathcal{H}' (1-3c_s^2) \Big) \pi^{(n)}   - c_s^2 {\nabla^2 \pi ^{(n)}}  
   %
   + (1-c_s^2)\Psi^{(n)} {\nabla^{2} \pi^{(n)}  }    
          \nonumber
     \\
      &
      %___________
    - 2 c_s^2  \Phi ^{(n)}  {\nabla^2 \pi^{(n)}}
          %___________
     + {3 c_s^2  \mathcal{H}^{(n)} (1+w) }\pi^{(n)} {\nabla^2 \pi^{(n)} }   
               %___________
     -  (1-c_s^2)
 \Big[ \frac{(\pi_{v  \; {i,j,k}}^{n+\frac{1}{2}} +\pi_{v \; {i,j,k}}^{n-\frac{1}{2}} )}{2}  + \mathcal{H} \pi^{(n)} \Big] {\nabla^2  \pi^{(n)}} 
        \nonumber
     \\
       &
               %___________
    - (2 c_s^2-1) {\nabla  \Psi^{(n)}  . \nabla \pi ^{(n)} }
            %___________
    + c_s^2 {\nabla  \Phi ^{(n)} . \nabla \pi^{(n)}  }                %___________
              +\frac{\mathcal{H}^{(n)}} {2 } \Big(2+3w+c_s^2  \Big) \,{\nabla  \pi^{(n)} . \nabla \pi^{(n)} }  
                      %___________
                           \nonumber
     \\
       &
          -2(1-c_s^2) \nabla  \pi^{(n)} .  \Big( \frac{{ \nabla  ( \pi_{v  \; {i,j,k}}^{n+\frac{1}{2}} +\pi_{v \; {i,j,k}}^{n-\frac{1}{2}} ) }  } {2}  + \mathcal{H}  \nabla\pi^{(n)} \Big) 
    \Big]
\end{align}
After some simplification we get,
\begin{align} 
%
 &\pi_v ^{n+\frac{1}{2}}= \frac{1}{1+ d\tau   \mathcal{H}^{(n)}  (1-3w) /2 - d\tau (1-c_s^2) \nabla^2 \pi^{(n)}/2} \times \Bigg[ \pi_v ^{n-\frac{1}{2}} - d \tau \Big [(1- 3w)\mathcal{H}^{(n)}   \frac{\pi_{v \; {i,j,k}}^{n-\frac{1}{2}} }{2}
     \nonumber
     \\
      &
  -3 { c_s^2 \mathcal{H}^{(n)}}\Big( 1- \frac{w}{c_s^2} \Big )\Psi^{(n) }
 - \, \frac{{\Psi}^{(n)}-{\Psi}^{(n-1)} }{d \tau}
      - 3  c_s^2  \, \frac{{\Phi}^{(n)}-{\Phi}^{(n-1)} }{d \tau}    
   +\Big( 3\mathcal{H}^2 (c_s^2 -w) + \mathcal{H}' (1-3c_s^2) \Big)\pi^{(n)}  
             \nonumber
     \\
      &
       - c_s^2 {\nabla^2 \pi ^{(n)}}  
   %
   + (1-c_s^2)\Psi^{(n)} {\nabla^{2} \pi^{(n)}  }    
      %___________
    - 2 c_s^2  \Phi ^{(n)}  {\nabla^2 \pi^{(n)}}
          %___________
     + {3 c_s^2  \mathcal{H}^{(n)} (1+w) }\pi^{(n)} {\nabla^2 \pi^{(n)} }   
          \nonumber
     \\
      &
               %___________
     -  (1-c_s^2)
 \Big( \frac{\pi_{v \; {i,j,k}}^{n-\frac{1}{2}} }{2} +\mathcal{H}  \pi^{(n)}  \Big) {\nabla^2  \pi^{(n)}} 
               %___________
    - (2 c_s^2-1) {\nabla  \Psi^{(n)}  . \nabla \pi ^{(n)} }
            %___________
    + c_s^2 {\nabla  \Phi ^{(n)} . \nabla \pi^{(n)}  }  
                              \nonumber
     \\
       & 
            %___________
              +\frac{\mathcal{H}^{(n)}} {2 } \Big(2+3w+c_s^2  \Big) \,{\nabla  \pi^{(n)} . \nabla \pi^{(n)} }  
                      %___________                    
         -2(1-c_s^2) \Big( \frac{{ \nabla  ( \pi_{v  \; {i,j,k}}^{n+\frac{1}{2}} +\pi_{v \; {i,j,k}}^{n-\frac{1}{2}} ) }  } {2}  + \mathcal{H}  \nabla\pi^{(n)} \Big) 
    \Big] \Bigg]
\end{align}
Since we have $\nabla \pi_v $ in the equation we use predictor corrector method as following,\\
In the first step we predict that the term $\nabla \pi \nabla \pi_v$ is small so we approximate $\nabla \pi_v ^{(n)}$ with $\nabla \pi_v^{n-\frac{1}{2}}$ then we calculate $\pi_v^{n+\frac{1}{2}}$ according to the formula with the guess, then we use the new $\pi_v^{n+\frac{1}{2}}$ into the full equations to correct $\pi_v^{n+\frac{1}{2}}$ . \\
 We have taken $\pi_{v  \; {i,j,k}}^{n} =\frac{(\pi_{v  \; {i,j,k}}^{n+\frac{1}{2}} +\pi_{v \; {i,j,k}}^{n-\frac{1}{2}} )}{2} $. Then we need to calculate $\mathcal{H}'$, ${\Psi}'$ and  ${\Phi}'$ in each loop, to calculate ${\Psi}'$ we save two $\Psi$ in each loop. \\
 On the other hand we have $\mathcal{H}'$ according to the Friedman equation, where we try to save $\mathcal{H}'$ from $a''$  and $\mathcal{H}$.
 %***************************
 %***************************
  %***************************
 %***************************
 %***************************
 %***************************
 \section{Stress tensor }
The most general action for a scalar field coupled to Einstein gravity is;
\be
S=\frac{1}{16 \pi G} \int \sqrt{-g} R d^4 x + \int \sqrt{-g} P (X, \varphi) d^4 x
\ee
The metric convention is $(-,+,+,+)$ and $X=- \frac{1}{2}  g^{\mu \nu}\partial _\mu \phi \partial_\nu \phi$. We  assume the scalar action as a matter sector which contributes to stress energy tensor,
\be
T^{\mu\nu}\equiv \dfrac {+2}{\sqrt {-g}}\dfrac {\delta \mathcal{ L}}{\delta g_{\mu\nu}}=\dfrac {2}{\sqrt {-g}}\dfrac {\delta \left[ \sqrt {-g}P\left( X,\varphi \right) \right] }{\delta g_{\mu\nu}}
=
\dfrac {2}{\sqrt {-g}}[- \dfrac {1 }{2 \sqrt{-g}} \frac{\delta g}{\delta g_{\mu\nu}}P\left( X,\varphi\right) +\dfrac {\delta P\left( X,\varphi\right) }{\delta g_{\mu\nu}}\sqrt {-g}]
\ee
According to appendix \ref{A1},
\be
\dfrac {\delta \sqrt {-g}}{\delta g_{\mu\nu}}=\dfrac {-1}{2\sqrt {-g}}\dfrac {\delta g}{\delta g_{\mu\nu}}=\dfrac {-1}{2\sqrt {-g}}\dfrac {g\delta g_{\mu\nu}g^{\mu\nu}}{\delta g_{\mu\nu}}=\dfrac {\sqrt {-g}}{2}g^{\mu\nu}
\ee
\be
T^{\mu\nu}=2\dfrac {\delta P\left( X,\varphi\right) }{\delta g_{\mu\nu}} + g^{\mu\nu}P\left( X,\varphi\right)
\ee
\be
T_{\rho \sigma}=g_{\mu \rho} g_{\nu \sigma} T^{\mu \nu}= \Big[ 2 g_{\mu \rho} g_{\nu \sigma}  \dfrac {\delta P\left( X,\varphi\right) }{-g_{\mu \rho'} g_{\nu \sigma'}  \delta g ^{\sigma' \rho'}} + g_{\mu \rho} g_{\nu \sigma}  g^{\mu\nu}P\left( X,\varphi\right) \Big]= -2\dfrac {\delta P \left( X,\varphi\right) }{\delta g^{\rho \sigma}}+g_{\rho \sigma}P\left( X,\varphi\right)
\ee
Where we have used $\delta g_{\mu \nu}= - g_{\mu \rho} g_{\nu \sigma} \delta g^{\rho \sigma}$.
\begin{align}
X=-\dfrac {1}{2}g^{\mu\nu}\partial_{\mu}\varphi\partial_{\nu}\varphi \longrightarrow  \delta X=-\dfrac {1}{2}\delta g^{\mu\nu}\partial_{\mu}\varphi\partial_{\nu}\varphi-\dfrac {1}{2}g^{\mu\nu}\partial_{\mu}\delta \varphi\partial_{\nu}\varphi-\dfrac {1}{2}g^{\mu\nu}\partial_{\mu}\varphi\partial_{\nu}\delta\varphi
\end{align}
so,
\be
\dfrac {\partial X}{\partial g^{\mu\nu}}=-\dfrac {\partial_{\mu}\varphi\partial_{\nu}\varphi}{2}
\ee
\be
\dfrac {\delta P}{\delta g^{\mu\nu}}=\dfrac {\partial P}{\partial X}\dfrac {\partial X}{\partial g^{\mu\nu}}+ \cancel{\dfrac {\partial P}{\partial\varphi}\dfrac {\partial\varphi}{\partial g^{\mu\nu}}}=\dfrac {\partial P}{\partial X}\dfrac {\partial X}{\partial g^{\mu\nu}}=-\dfrac {\partial_{\mu}\varphi\partial_{\nu}\varphi}{2}P_{,X}
\ee
\be
T_{\mu\nu}=g_{\mu\nu}P\left( X,\varphi\right) +P_{,X}\partial_{\mu}\varphi\partial_{v}\varphi \; , \;
T_{\mu\nu}=\left( \rho+p\right) u_{\mu}u_{\nu}+p g_{\mu\nu}
\ee
\be
u_{\mu}=\dfrac {\partial_{\mu}\varphi}{\sqrt {-\partial_{\mu}\varphi\partial^{\mu}\varphi}}\rightarrow u_{\mu}=\dfrac {\partial_{\mu}\varphi}{\sqrt {2X}} , \rho=2XP_{,X}-P \; , \; p=P  \label{eq10}
\ee
We assume that field is a monotonic function of time in background which is perturbed in each constant physical time hypersurfaces. \\
It is important to notice that in the previous chapter $\pi$ was the perturbation in constant physical time hypersurfaces, so to be consistent,  although at the we want to express everything in terms of conformal time in Gevolution but we keep $\pi$ as perturbation in physical time.
\be
\varphi_{0}\left( \tau+\pi\left( \tau,\overrightarrow {x}\right) \right) =\varphi_{0}\left( \tau \right) +\dfrac {\partial\varphi_{0}}{\partial  \tau }\pi+\dfrac {\partial^{2}\varphi_{0}}{2\partial^{2} \tau}\pi^{2}+\ldots
\ee
We can choose $\varphi_0(\tau)=\tau$ for simplicity, using the following ansatz for the metric,
\be
g_{\mu\nu}=a(\tau)^2 \Big [-e^{2\Psi}d\tau^{2}+ e^{-2\Phi}dr^{2} \Big]
\ee
where $\tau$ is the conformal time.
\be
\delta g^{(1)}_ { 00}=-2\, a^2 \Psi \, \; \; , 
\delta g^{(1)}_{ij}= -2 a^{2} \Phi \delta_{ij}
\ee
Where $\delta g^{(1)}_ { 00}$ means the first order metric in pertubations.  The inverse of metric is defined as following,
\be
g^{\mu\nu}=\frac{1}{a^2} \Big [-e^{-2\Psi}d\tau^{2}+e^{2\Phi}dr^{2}  \Big ]
\ee
\be
\delta g_{(1)}^{00}=+\frac{2\Psi}{a^2} \, \; \; , 
\delta g_{(1)}^{ij}= + \frac{2\Phi \delta^{ij} }{a^2}
\ee
We have,
\be
X=\dfrac {-1}{2}g^{\mu\nu} (\tau + \pi,x)\partial_{\mu}\left( \tau+\pi\right) \partial_{\nu}\left( \tau+\pi\right) 
\ee
We expand X perturbatively,
\be
X=\overline {X}+\delta X_{1}+ \delta X_{2}+\ldots
\ee
\be
\overline {X}=-\dfrac {1}{2}\bar{g}^{00}\partial_{0} \tau \partial_{0} \tau=+\dfrac {1}{2 a^2}\\
\ee
\be
\delta X_{1}={\bar{X}'} \pi-\dfrac {1}{2}\delta g_{(1)}^{00}\partial_{0} \tau \partial_{0} \tau-\dfrac {1}{2} \bar{g}^{00}\partial_{0} \tau \partial_{0}\pi-\dfrac {1}{2} \bar{g}^{00}\partial_{0}\pi\partial_{0} \tau-\dfrac {1}{2}\bar{g}^{ij}\partial_{i}\pi\partial_{j}\pi
\ee
where ${\bar{X}'} \pi =-\dfrac {1}{2}\bar{g}^{00'} \pi \partial_{0} \tau \partial_{0} \tau = -\frac{\mathcal{H}}{a^2} \pi $.
\be
 \delta X_{1}=\frac{1}{a^2} \Big [- \mathcal{H} \pi-\Psi+{\pi'}- \frac{(\vec{\nabla} \pi)^2}{2 }  +O\left( \varepsilon^{2}\right) \Big]
\ee
We do not need to calculate $X_2$ since the energy momentum constraint adds at most one spatial derivative which does not add the second order terms to first order. So
\bea
 & P\left(\varphi_0( \tau+\pi) ={\varphi_0}+ {{\varphi_0}}' \pi,\overline{X}+ \delta X_1+\delta X_2 \right)  = \overline{P}\left( \varphi( \tau),\overline {X}\right) 
+ {\dfrac {\partial\overline {P}}{\partial \varphi_0}} \pi+\dfrac {1}{2} {\dfrac {\partial^{2}\overline {P}}{\partial \varphi_0^2}}\pi^{2}
+
\nonumber \\ &
\dfrac {\partial\overline {P}}{\partial\overline {X}}\delta X_1+\dfrac {1}{2}\dfrac {\partial P}{\partial X^{2}}\delta X_1^{2}+\dfrac {\partial^2 P}{\partial X \partial \varphi_0}\delta X_1 \pi +\dfrac {1}{2}\dfrac {\partial P}{\partial X^{2}}\delta X_2 + \mathcal{O}(\epsilon^3).
\eea
\\
 Note that here $\pi$ is perturbation in $\tau$ conformal time.%The term $\dfrac {\partial\overline {P}}{\partial \tau}$ becomes  $\dfrac {\partial\overline {P}}{\partial \tau}=\dfrac {\partial\overline {P}}{\partial  {\varphi}} \dfrac{\partial {\varphi}}{\partial \tau}+ \dfrac {\partial\overline {P}}{\partial  {X}} \dfrac{\partial {X}}{\partial \tau} =P_{,\varphi} \dot{\bar{\varphi}} + \bar{P}_{\bar{X}}$. Because $\varphi$ and $\partial_{\mu} \varphi$ are independent variables not function of $\tau$. \\
The adiabatic sound speed ({\color{red}why?!}) is defined as below,
\be
%c^{2}_{s}\equiv \frac{\delta P}{\delta \rho} =\dfrac {\bar{P}_{,X} \delta X + \bar{P}_{,\varphi} \delta \varphi}{\bar{\rho}_{,X} \delta X  +\bar{\rho}_{,\varphi} \delta  \varphi}=\dfrac {\bar{P}_{,X}}{\bar{\rho}_{,X}}=\dfrac {\bar{P}_{,X}}{\bar{P}_{,X}+2\bar{X}\bar{P}_{,XX}} 
c^{2}_{s}\equiv \dfrac {\bar{P}_{,X}}{\bar{\rho}_{,X}}=\dfrac {\bar{P}_{,X}}{\bar{P}_{,X}+2\bar{X}\bar{P}_{,XX}} 
\ee
and $\Omega$
\be
\Omega= \frac{\bar{\rho}}{3 M_{pl}^2 H^2}= \frac{ a^2 \bar{\rho}}{3 M_{pl}^2 \mathcal{H}^2}=\frac{{2\bar{X} \bar{P}_{,X}-\bar{P}}}{3 M_{pl}^2 H^2} \label{22}
\ee
Where we have used $ \rho=2XP_{,X}-P$
\be
\omega=\dfrac {\overline {P}}{\overline {\rho}}=\dfrac {\overline {P}}{2\overline {X} \, \overline{P}_{,X}-\overline {P}} \label{23}
\ee
Moreover we have,
\be
\rho'=2X' P_{,X} + 2 X P_{,X \varphi} \varphi' +2 X P_{,X X} X' - P_{,\varphi} \varphi' - P_{,X} X'= (2 X P_{,X \varphi} - P_{,\varphi}) \varphi' + (P_{,X} + 2 X P_{,XX})X'
\ee
In the background level and using $\bar{X}=\frac{1}{2 a^2}$, $\bar{X}'=-\frac{\mathcal{H}}{a^2}$, $\varphi'=1$
\be
\rho' =\frac{P_{,X}'}{a^2} - P'  - (P_{,X} + \frac{ P_{,XX}}{a^2}) \frac{\mathcal{H}}{a^2}
\ee
So we can write the function $P$ and it derivative in terms of $\Omega$, $\omega$ and $c_s^2$,
\be
\bar{P}_{X}= a^2 \bar{P} (1+\frac{1}{\omega}) \; \; \; \; \;  \; \bar{P} _{,XX}=a^2  \bar{P}_{,X} \frac{1-c_s^2}{c_s^2} =a^4  \bar{P} (1+\frac{1}{\omega}) (\frac{1}{c_s^2} -1 )
\label{Pbarder}
\ee
So according to \ref{22} and \ref{23}\\
\be
\bar{P}=  3 M_{pl}^2 H^2 \Omega \, \omega = \frac{ 3 M_{pl}^2 \mathcal{H}^2 \Omega\, \omega }{a^2} \label{Pbar}
\ee
Moreover,
\be
\frac{\bar{P}'}{\bar{P}}=\frac{2 \mathcal{H}' }{\mathcal{H}} + \frac{\Omega'}{\Omega} + \cancel{\frac{w'}{w} }- 2\mathcal{H}
\ee
%Where according to continuity equation \ref{Conteqgg}, we can write,
%\be
%\frac{\bar{P}'}{\bar{P}}=-3 (1+w) \mathcal{H}- \mathcal{H}
%\ee
%where we have assumed that $w'=0$ and we have used the below relation,
%\be
% \frac{\Omega'}{\Omega}=- \frac{(1+3  w)}{2} \mathcal{H}- \frac{ \mathcal{H'}}{\mathcal{H}}
%\ee
Also we have the relation for $\bar{P}'_{,X}$ as following,
\be
\frac{\bar{P}'_{,X}}{\bar{P}_{,X}}=\frac{\bar{P}'}{\bar{P}}+2 \mathcal{H}
\ee
So we have,
\begin{align}
 -a^2 \bar{P}'  + \bar{P}'_{,X}&=-a^2 \bar{P}' + a^2 \bar{P} (1+\frac{1}{w}) (\frac{\bar{P}'}{\bar{P}}+2 \mathcal{H}) = a^2 \frac{\bar{P}'}{w} + 2 \mathcal{H} a^2 \bar{P} (1+\frac{1}{w})
 \\ \nonumber&
 = \frac{a^2 \bar{P}}{w} \Big[ 2 \frac{\mathcal{H}'}{\mathcal{H}}+ \frac{\Omega'}{\Omega} + 2 \mathcal{H} w\Big]
\end{align}
It is very important to note that by $\bar{P}(\varphi,X)'$ we mean $\bar{P}(\varphi,X)_{,\varphi}$ and $\bar{P}_{,\tau}=\bar{P}_{,\varphi} \varphi' + \bar{P}_{,X} X' $
The relation between Hubble and conformal Hubble is;
\be
\mathcal{H} (\tau)=\frac{1}{a(\tau) }\frac{d a(\tau)}{d \tau }= \frac{1}{a(t) } \frac{d a (t) }{d t} \frac{d t }{ d\tau}= a H(t)
\ee
and for the derivative,
\begin{align}
& \dot{H}= \frac{-\mathcal{H}^2+ \mathcal{H}'}{a^2} \nonumber \\ &
\mathcal{H}'=a^2 \Big[ H^2 + \dot{H}\Big ]
\end{align}
Now we can construct stress tensor up to first order in Gevolution's perturbation scheme,
\begin{align} \label{eqTmunuI}
T_{\mu \nu} &= P g_{\mu \nu} + P_{,X} \partial_{\mu} \varphi \partial_{\nu} \varphi
 \\
  \nonumber
   & =(\bar{g}_{\mu \nu} + \delta g^{(1)}_{\mu \nu}) (\bar{P}+\bar{P}' \pi+\bar{P}_{,X} \delta X_1) + (\bar{P}_{,X}+\bar{P}'_{,X} \pi+\bar{P}_{,XX} \delta X_1) \partial_{\mu} (\tau+ \pi) \partial_\nu (\tau+\pi)+ \ldots
\\ \nonumber & 
= \Big[ \bar{g}_{\mu \nu} \bar{P} 
+
 \bar{P}_{,X} \partial_{\mu} \tau \partial_{\nu} \tau \Big] \epsilon ^0 
+
\Big[( \bar{g}_{\mu \nu} \bar{P}'+ \bar{P}'_{,X} )\pi+\bar{g}_{\mu \nu}  \bar{P}_{,X} \delta X_1 
+
 \delta g^{(1)}_{\mu \nu} \bar{P} 
 +
  \bar{P}_{,X}  \left ( \partial_{\mu} \pi \partial_{\nu} \tau  
  +
  \partial_{\mu} \tau \partial_{\nu} \pi  \right ) 
    \nonumber \\ &
  +
   \delta X_1 \bar{P}_{,XX}   \partial_{\mu} \tau \partial_{\nu} \tau  
      +
    \bar{P}_{,X}   \partial_{\mu} \pi \partial_{\nu} \pi
    +  \delta X_1 {\color{red} \bar{P}_{,XX}  \big(  \partial_{\mu} \tau \partial_{\nu} \pi  +   \partial_{\mu} \pi \partial_{\nu} \tau  \big) }
    + \cancelto{\mathcal {O}(\epsilon^{2})}{\delta X_1 \bar{P}_{,XX}   \partial_{\mu} \pi \partial_{\nu} \pi  
   } \Big ] 
 \; \; \;+ \mathcal {O}(\epsilon^{2}) 
\end{align}
It is important to know that $\bar{g}_{\mu \nu}' \bar{P}=0$ since the metric is not a function of scalar field, so it is no affected by changing the scalar field which is a dynamical variable. 
{\color{red}The red color is the one I have tension with Filippo's calculation.}
%\newpage
\subsection{$T_{00}$ component of energy momentum tensor}
According to previous equation $T_{00} $ component is,
\begin{align} \label{T00}
T_{00} &=
   \Big[ \bar{g}_{0 0} \bar{P} 
+
 \bar{P}_{,X} \partial_{0} \tau \partial_{0} \tau \Big] \epsilon ^0 
+
\Big[( -a^2 \bar{P}' +\bar{P}'_{,X} )\pi+\bar{g}_{0 0}  \bar{P}_{,X} \delta X_1 
+
 \delta g^{(1)}_{0 0} \bar{P} 
 +
  \bar{P}_{,X}  \left ( \partial_{0} \pi \partial_{0} \tau  
  +
  \partial_{0} \tau \partial_{0} \pi  \right ) 
    \nonumber
 \\
  &
  +
   \delta X_1 \bar{P}_{,XX}   \partial_{0} \tau \partial_{0} \tau 
   +
    \cancelto{\mathcal {O}(\epsilon^{2}) 
} { \bar{P}_{,X}  } \partial_{0} \pi \partial_{0} \pi  +   \cancelto{\mathcal {O}(\epsilon^{2})}  {2 \pi'   } \delta X_1  \bar{P}_{,XX}  \Big ]
%::::::::::::::::::::::::::::::::::::::::::::::::::::::::::::::::::::::::::::::::::::::::::::::::::::::::::::::::
%::::::::::::::::::::::::::::::::::::::::::::::::::::::::::::::::::::::::::::::::::::::::::::::::::::::::::::::::
  \nonumber
 \\
  &
  =
  [-a^2  \bar{P} 
+
 \bar{P}_{,X}  ] \epsilon ^0 
+
\Big[a^2 (- \bar{P}' +\frac{\bar{P}'_{,X}}{a^2} )\pi- a^2 \bar{P}_{,X} \delta X_1 
-
 2 a^2 \Psi \bar{P} 
 +
 2 \bar{P}_{,X}   {\pi'}
  +
    {\color{red}
   \delta X_1 \bar{P}_{,XX} }
  \Big ] 
+ \mathcal {O}(\epsilon^{2}) 
%::::::::::::::::::::::::::::::::::::::::::::::::::::::::::::::::::::::::::::::::::::::::::::::::::::::::::::::::
%::::::::::::::::::::::::::::::::::::::::::::::::::::::::::::::::::::::::::::::::::::::::::::::::::::::::::::::::
  \nonumber
 \\
  &
  =
  [-a^2 \bar{P} 
+
a^2 \bar{P}  (1+\frac{1}{w}) ] \epsilon ^0 
+
\Big[a^2 \rho' \pi
-
 2 a^2  \Psi \bar{P} 
 +
 2  a^2 \bar{P}  (1+\frac{1}{w})  {\pi'}
  +
  a^4 \bar{P}  (1+\frac{1}{w}) (\frac{1}{c_s^2}-1) \,   (\delta X_1 )
   \Big ] 
+ \mathcal {O}(2) 
%::::::::::::::::::::::::::::::::::::::::::::::::::::::::::::::::::::::::::::::::::::::::::::::::::::::::::::::::
%::::::::::::::::::::::::::::::::::::::::::::::::::::::::::::::::::::::::::::::::::::::::::::::::::::::::::::::::
  \nonumber
 \\
  &
  =
 \frac{ a^2 \bar{P}}{w} \,\epsilon ^0 
+
\frac{ a^2\bar{ P}}{w}   \Big[ -3 \mathcal{H} (1+w) \pi
-
 2   w \Psi
 +
 2  (1+w)  {\pi'}
  +
  a^2 (1+w) (\frac{1}{c_s^2}-1) \, \Big(  -\Psi+{\pi'}-  \frac{(\vec{\nabla} \pi)^2}{2} \Big)
   \Big ] \epsilon^1
%+ \mathcal {O}(\epsilon^{3/2}) 
%::::::::::::::::::::::::::::::::::::::::::::::::::::::::::::::::::::::::::::::::::::::::::::::::::::::::::::::::
%::::::::::::::::::::::::::::::::::::::::::::::::::::::::::::::::::::::::::::::::::::::::::::::::::::::::::::::::
%  \nonumber
% \\
%  &
%  =
%\frac{a^2 \bar{ P}}{w} \,\epsilon ^0 
%+
%\frac{a^2 \bar{ P}}{w}   \Big[ (2 \frac{\mathcal{H}'}{\mathcal{H}}+ \frac{\Omega'}{\Omega}  ) \pi+  a^2 (1+w) (\frac{1}{c_s^2}- 2)  \delta X_1 
%-
% 2 w \Psi
% +
% 2  (1+w)  {\pi'}
% \Big ] 
%+ \mathcal {O}(\epsilon^{2}) 
%%::::::::::::::::::::::::::::::::::::::::::::::::::::::::::::::::::::::::::::::::::::::::::::::::::::::::::::::::
%%::::::::::::::::::::::::::::::::::::::::::::::::::::::::::::::::::::::::::::::::::::::::::::::::::::::::::::::::
%  \nonumber
% \\
%  &  
%  =
%3 M_{pl}^2 \mathcal{H}^2 \Omega  \,\epsilon ^0 
%+
%3 M_{pl}^2  \mathcal{H}^2 \Omega    \Bigg[ (2 \frac{\mathcal{H}'}{\mathcal{H}}+ \frac{\Omega'}{\Omega}  ) \pi+ (1+w) (\frac{1}{c_s^2}- 2)  \Big[\mathcal{H} \pi -\Psi+{\pi'}-  \frac{(\vec{\nabla} \pi)^2}{2} \Big ]
%-
%  \nonumber
% \\
%  &
% 2 w \Psi
% +
% 2  (1+w)  {\pi'}
% \Bigg ]
%+ \mathcal {O}(\epsilon^{2})  
%%::::::::::::::::::::::::::::::::::::::::::::::::::::::::::::::::::::::::::::::::::::::::::::::::::::::::::::::::
%%::::::::::::::::::::::::::::::::::::::::::::::::::::::::::::::::::::::::::::::::::::::::::::::::::::::::::::::::
%  \nonumber
% \\
%  &
%  =
%3 M_{pl}^2  \mathcal{H}^2 \Omega \Bigg[  1 +(2 \frac{\mathcal{H}'}{\mathcal{H}}+ \frac{\Omega'}{\Omega}  ) \pi+ \Psi \Big (- (1+w) (\frac{1}{c_s^2}- 2)-2 w  \Big ) + {\pi'} \Big ( (1+w) (\frac{1}{c_s^2}- 2 )+2 (1+w)   \Big) 
%   \nonumber
% \\
%  & - \frac{(\vec{\nabla} \pi)^2}{2}  \Big ( (1+w) (\frac{1}{c_s^2}- 2 )  \Big )
% \Bigg]
 %::::::::::::::::::::::::::::::::::::::::::::::::::::::::::::::::::::::::::::::::::::::::::::::::::::::::::::::::
%::::::::::::::::::::::::::::::::::::::::::::::::::::::::::::::::::::::::::::::::::::::::::::::::::::::::::::::::
   \nonumber
 \\
  &
  =
3 M_{pl}^2  \mathcal{H}^2 \Omega \Bigg[  1-3\mathcal{H} (1+w)  \pi+ \Psi \Big (2 - \frac{1+w}{c_s^2}  \Big ) + {\pi'} \Big ( \frac{1+w}{c_s^2}   \Big)  - \frac{(\vec{\nabla} \pi)^2}{2}   (1+w) (\frac{1}{c_s^2}- 2 ) 
 \Bigg]+  \mathcal {O}(\epsilon^{2}) 
\end{align}
Where we have used \ref{Pbarder} and \ref{Pbar}.
Finaly the $T_{00}$ component is;
%\begin{empheq}[box=\mymath ]{equation*}
%If we convert the $\pi$ to be the perturbation in constant physical time hypersurfaces (not constant conformal time) we get,
%$\pi_{phys}= \pi_{conf}$
\begin{empheq}[box=\mymath ]{equation*}
\begin{align}
T_{00}=  
3 M_{pl}^2   \mathcal{H}^2\Omega \Bigg[  1-3\mathcal{H} (1+w)  \pi+  \Psi \Big (2 - \frac{1+w}{c_s^2}  \Big ) + {\color{blue} ({\pi'}+ \mathcal{H} \pi) } \Big ( \frac{1+w}{c_s^2}   \Big)  -   \frac{(\vec{\nabla} \pi)^2}{2}    (1+w) (\frac{1}{c_s^2}- 2 ) 
 \Bigg]+  \mathcal {O}(\epsilon^{2}) 
\end{align}
\end{empheq}
Now we compare the result with equation 147 of https://arxiv.org/pdf/1411.3712.pdf:\\
It is clear that the coefficient of $\pi$ is the same, but we should remember that we changed a sign to get this result ($\rho'$) and also $H \pi_{phys}$ = $\mathcal{H} \pi_{\text{conf}}$. \\
${\color{blue} \dot{\pi_{phys}} = \partial_t(a \pi_{conf})=(a \dot{\pi_{con}} + aH \pi_{con} -\Psi)= \pi_{con}'+\mathcal{H} \pi_{con} -\Psi}$ \\
The other terms in the paper are:
\be
 H^2 \alpha_k \mathcal{P}= \frac{\bar{\rho } (1+w)}{c_s^2}(\dot{\pi}- \Psi)=  \frac{\bar{\rho } (1+w)}{c_s^2}({\pi'}_{\text{conf}}- \Psi)
\ee
$\Psi$ here means $\Phi$ in the paper. Moreover $\dot{\pi}_{phys}=\pi_{conf}'$. The extra $2 \Psi$ term here comes from the fact that $T_{00}=g_{00} T^0_0$. \\
In sum, up to first order we get the same equatuin as the references. Moreover the extra $a^2$ factor is in $\Omega =  \frac{a^2 \bar{\rho}}{3 M_{pl}^2 \mathcal{H}^2}$
\subsection{$T_{0i}$ component of energy momentum tensor}
According to the equation \ref{eqTmunuI} we can calculate $T_{0i}$ component of energy momentum tensor
\begin{align}
T_{\mu \nu}    & =(\bar{g}_{\mu \nu} + \delta g^{(1)}_{\mu \nu}) (\bar{P}+\bar{P}' \pi+\bar{P}_{,X} \delta X_1) + (\bar{P}_{,X}+\bar{P}'_{,X} \pi+\bar{P}_{,XX} \delta X_1) \partial_{\mu} (\tau+ \pi) \partial_\nu (\tau+\pi)+ \ldots
%\label{eqTmunu}
\end{align}
So $T_{0i}$ reads;
\begin{align} \label{T0i}
T_{0 i} &
= \Big[\cancel{\bar{g}_{0 i}} \bar{P} 
+
 \bar{P}_{,X} \partial_{0} \tau \partial_{i} \tau \Big] \epsilon ^0 
+
\Big[ \cancel{\bar{g}_{0 i}} \bar{P}' \pi+ \cancel{\bar{g}_{0 i}}  \bar{P}_{,X} \delta X_1 
+
 \delta g^{(1)}_{0 i} \bar{P} 
 +
  \bar{P}_{,X}  \Big( \partial_{0} \pi \cancel{ \partial_{i} \tau  }
  +
  \partial_{0} \tau \partial_{i} \pi  \Big ) 
      \nonumber  \\&
  +
   \delta X_1 \bar{P}_{,XX}   \partial_{0} \tau  \cancel{\partial_{i} \tau  }
   +
     \delta X_1 \bar{P}_{,XX}   \partial_{0} \tau  \partial_{i} \pi
  +
    {\bar{P}_{,X} } \partial_{0} \pi \partial_{i} \pi \; \;   \Big ]
+ \mathcal {O}(\epsilon^{2}) 
\nonumber 
\\ 
&
= [0] \epsilon ^0 
+
\Big[
 \delta g^{(1)}_{0 i} \bar{P} 
 +
  \bar{P}_{,X}   \partial_{0} \tau \partial_{i} \pi +  {\bar{P}_{,X} } \partial_{0} \pi \partial_{i} \pi + \delta X_1 \bar{P}_{,XX}    \partial_{i} \pi
  \Big ] 
+ \mathcal {O}(\epsilon^{2}) 
\nonumber 
\\ 
&
= 
3 M_{pl}^2 \mathcal{H}^2 \Omega \Bigg[
 w \, \frac{\delta g^{(1)}_{0 i} }{a^2}
 +
   (1+w )\,  \partial_{i} \pi + (1+w) \, \pi' \partial_{i} \pi  -(1+ w)(\frac{1}{c_s^2}-1)  \frac{(\vec{\nabla} \pi)^2}{2} \partial_{i} \pi 
   \Bigg ]
+ \mathcal {O}(\epsilon^{2}) 
%\label{eqTmunu}
\end{align}
If we neglect vector perturbation and keeping only short wave correction in first order we have,
\begin{empheq}[box=\mymath]{equation*}
\begin{align}
T_{0i}= 
3 M_{pl}^2 \mathcal{H}^2 \Omega \Bigg[
    (1+w )\,  \partial_{i} \pi -(1+ w)(\frac{1}{c_s^2}-1)   \frac{(\vec{\nabla} \pi)^2}{2} \partial_{i} \pi 
   \Bigg ]
+ \mathcal {O}(\epsilon^{2}) 
\end{align}
\end{empheq}
Note that $T_{0i}=T_{i0}$. To first order the followed expression is the same as equation 148 which is $q_D=-\bar{\rho} (1+w) \pi$, which means $T_{0i} = -\bar{\rho} (1+w) \partial_i \pi/a = -\bar{\rho} (1+w) \partial_i \pi_{\text{conf}} $, since $T_{0i}=g_{00}T^0_i=- T^0_i$ and $g_{00}$ in the paper is "-1".
\subsection{$T_{ij}$ component of energy momentum tensor}
It is noteworthy to mention that by $P'$ we mean $P_{\varphi}$ while the expression for $P_{\tau}= P_{\varphi} \varphi' + P_{X} X'$ since P is a function of $\varphi$ and $X$.
Again using the equation \ref{eqTmunuI}
\begin{align}
T_{\mu \nu} &
= \Big[ \bar{g}_{\mu \nu} \bar{P} 
+
 \bar{P}_{,X} \partial_{\mu} \tau \partial_{\nu} \tau \Big] \epsilon ^0 
+
\Big[ ({\bar{g}_{\mu \nu}} \bar{P}' +\bar{P}'_{, X})\pi+ \bar{g}_{\mu \nu}  \bar{P}_{,X} \delta X_1 
+
 \delta g^{(1)}_{\mu \nu} \bar{P} 
 +
  \bar{P}_{,X}  \left ( \partial_{\mu} \pi \partial_{\nu} \tau 
  +
  \partial_{\mu} \tau \partial_{\nu} \pi  \right ) 
       \nonumber \\ &
  +
   \delta X_1 \bar{P}_{,XX}   \partial_{\mu} \tau \partial_{\nu} \tau  
   +
    \bar{P}_{,X}   \partial_{\mu} \pi \partial_{\nu} \pi  + 
    +  \delta X_1 \bar{P}_{,XX}  \big(  \partial_{\mu} \tau \partial_{\nu} \pi  +   \partial_{\mu} \pi \partial_{\nu} \tau \big ) \Big]
+ \mathcal {O}(\epsilon^{3/2}) 
%\label{eqTmunu}
\end{align}.
So $T_{ij}$ is;
\begin{align} \label{Tij}
T_{i j} &
= \Big[ \bar{g}_{i j} \bar{P} 
+
 \bar{P}_{,X} \cancel{ \partial_{i} \tau}\partial_{j} \tau \Big] \epsilon ^0 
+
\Big[ ({\bar{g}_{i j}} \bar{P}' \pi+\cancel{\bar{P}'_{, X}} \partial_{i} (\tau+\pi) \partial_{j}(\tau+\pi))\pi+ \bar{g}_{i j}  \bar{P}_{,X} \delta X_1 
+
 \delta g^{(1)}_{i j} \bar{P} 
        \nonumber \\ &
 +
  \bar{P}_{,X}  \left ( \partial_{i} \pi \cancel{\partial_{j} \tau  }
  +
\cancel{  \partial_{i} \tau }\partial_{j} \pi  \right ) 
  +
   \delta X_1 \bar{P}_{,XX}   \cancel{ \partial_{i} \tau }\partial_{j} \tau  
   +
    \bar{P}_{,X}   \partial_{i} \pi \partial_{j} \pi \Big ]
+ \mathcal {O}(\epsilon^{2}) 
\nonumber \\ & 
%::::::::::::::::::::::::::::::::::::::::::::::::::::::::::::::::::::::::::::::::::::::::::::::::::::::::::::::::
%::::::::::::::::::::::::::::::::::::::::::::::::::::::::::::::::::::::::::::::::::::::::::::::::::::::::::::::::
= \Big[ a^2 \delta_{ij} \bar{P} 
 \Big] \epsilon ^0 
+
\Big[ a^2   (\delta_{ij}  \bar{P}' + P_{,X} \bar{X}' )\pi+ a^2  \delta_{ij}   \bar{P}_{,X} \delta X_1 
-
 2 a^2 \Phi \delta_{ij} \bar{P} 
     +
    \bar{P}_{,X}   \partial_{i} \pi \partial_{j} \pi \Big ] 
+ \mathcal {O}(\epsilon^{2}) 
\nonumber \\ & 
%::::::::::::::::::::::::::::::::::::::::::::::::::::::::::::::::::::::::::::::::::::::::::::::::::::::::::::::::
%::::::::::::::::::::::::::::::::::::::::::::::::::::::::::::::::::::::::::::::::::::::::::::::::::::::::::::::::
= \frac{ a^2 \bar{P}}{w}   \Big[ w \delta_{ij} 
\Big] \epsilon ^0 
+
\frac{ a^2  \bar{P}}{w} \Big[ w   \delta_{ij}  \bar{P}_{,\tau}  /\bar{P} \pi+  \delta_{ij}   (1+w)  \Big (-{\mathcal{H}} \pi-\Psi+{\pi'}- \frac{ (\vec{\nabla} \pi)^2 }{2 } \Big ) 
-
 2  w \Phi \delta_{ij} 
        \nonumber \\ &
     +
   (1+w)  \partial_{i} \pi \partial_{j} \pi \Big ] 
+ \mathcal {O}(\epsilon^{2}) 
\nonumber \\ & 
%::::::::::::::::::::::::::::::::::::::::::::::::::::::::::::::::::::::::::::::::::::::::::::::::::::::::::::::::
%::::::::::::::::::::::::::::::::::::::::::::::::::::::::::::::::::::::::::::::::::::::::::::::::::::::::::::::::
=3 M_{pl}^2 \mathcal{H}^2 \Omega    \Bigg[w \delta_{ij} -3 \mathcal{H} w (1+w)\pi \delta_{ij} 
+
 \delta_{ij}   (1+w)  \Big (-\Psi+{\pi'}- \frac{(\vec{\nabla} \pi)^2  }{2 }  \Big )
-
 2 w \Phi \delta_{ij} 
     +
   {(1+w)}  \partial_{i} \pi \partial_{j} \pi \Bigg ]
          \nonumber \\ & 
%:::::::::::::::::::::::::::::::::::::::::::::::::::::::::::::::::::::::::::::::::::::::::::::::::::::::::::::::: 
%:::::::::::::::::::::::::::::::::::::::::::::::::::::::::::::::::::::::::::::::::::::::::::::::::::::::::::::::
\end{align}
As a result we can write;
\begin{empheq}[box=\mymath]{equation*}
\begin{align}
T_{ij}=&3 M_{pl}^2 \mathcal{H}^2 \Omega  \Bigg[  w \delta_{ij}  -3 \mathcal{H} w (1+w)\pi \delta_{ij}
-
 2 w \Phi \,  \delta_{ij} + (1+w) ( {\color{blue} ({\pi'}+ \mathcal{H} \pi) }-\Psi)  \, \delta_{ij}
 \nonumber  \\ &
 - \frac{1+w}{2 }   (\vec{\nabla} \pi)^2 \, \delta_{ij} +({1+w}  )\partial_{i} \pi \partial_{j} \pi    \Bigg ] 
     + \mathcal {O}(\epsilon^{2}) 
\end{align}
\end{empheq}
Comparing with equation 150 of https://arxiv.org/pdf/1411.3712.pdf:
\be
T_{ij}^{(paper)}= g_{ii} T^i_{j} =a^2 \Big(\dot{p} \pi + \frac{\rho_D + p_D}{M^2} \times M^2( \dot{\pi} -\Psi) \Big) \delta_{ij} = \delta_{ij} a^2[-3 \frac{\mathcal{H}}{a} \bar{\rho} (1+w) a \pi_{\text{conf}} + \bar{\rho} (1+w) ( {\pi}'_{\text{conf}} -\Psi))]
\ee
Which is the same  and extra $a^2$ coefficient in their paper  is in the definition of $\Omega$ which is defined by physical Hubble constant! Moreover extra term -2$w \Phi$ comes from $\delta g_{ik} T^{k}_j$
The diagonal components read; 
\be
T_{ii}=3 M_{pl}^2 \mathcal{H}^2 \Omega   \Bigg[   w -3 \mathcal{H} w (1+w) \pi
-
 2 w  \, \Phi  - (1+w)  \, \Psi +  (1+w) \,  {\pi'} 
 + \frac{1+w}{2 }    (\vec{\nabla} \pi)^2    \Bigg ] 
     + \mathcal {O}(\epsilon^{3/2}) 
\ee
The off diagonal components are; 
\be
T_{ij}=3 M_{pl}^2 \mathcal{H}^2 \Omega (1+w)   \,   \partial_{i} \pi \partial_{j} \pi   
     + \mathcal {O}(\epsilon^{3/2}) 
\ee
\section{Field transfer function}
Before trying to get the field transfer function we try to do some tests on different results in order to catch all the physics we are dealing with.

\section{Gevolution implementation }
In Gevolution we should use $ M^2_{pl}= 1/8 \pi G$.\\
%And it seems the normalization factor is $-3 \mathcal{H}_0^2T_0^0/8\pi G$ \\
So we have:
%\begin{empheq}[box=\mymath]{equation}
\begin{align}
T_{0i}= 
3 M_{pl}^2 \mathcal{H}^2 \Omega \Bigg[
    (1+w )\,  \partial_{i} \pi -(1+ w)(\frac{1}{c_s^2}-1)   \frac{(\vec{\nabla} \pi)^2}{2} \partial_{i} \pi 
   \Bigg ]
+ \mathcal {O}(\epsilon^{2}) 
\end{align}

\begin{align}
 & T_0^0 (Gev)=-a^3 {T_{0}^{0}}=   {3 a M_{pl}^2   \mathcal{H}^2\Omega} \Bigg[1+ \frac{1+w}{c_s^2} \Big(- 3 \mathcal{H}c_s^2 \pi- \Psi+  {\pi'}  -  \Big(1-2 c_s^2 \Big) 
 \frac{(\vec{\nabla} \pi)^2}{2} \Big )   \Bigg ]
\nonumber \\ &
T^{0}_{i}(Gev)=a^3 T^0_i = -a T_{0i} = -{3 a M_{pl}^2   \mathcal{H}^2\Omega}  (1+w)\Big[1 - (\frac{1}{c_s^2} -1)  \frac{(\vec{\nabla} \pi)^2}{2}  \Big ] \partial _i \pi 
\nonumber \\ &
T_{j}^{i}(Gev)= a^3 T_j^i = {3 a M_{pl}^2   \mathcal{H}^2\Omega w} \Bigg ( 1+  \frac{1+w}{w}\Big [ -3 \mathcal{H} w \pi- \Psi + \pi' -  \frac{(\vec{\nabla} \pi)^2}{2}   \Big] \delta_{j}^{i}  + \frac{1+w}{w} \delta^{i k} \partial_k \pi \partial_j \pi  \Bigg) 
\end{align}
%\end{empheq}
Note that $\dot{H}$ is determined by all the matter contents of the universe not by k-essence alone, the continuity equation for k-essence or matter gives the dynamics of density. \\
About the unit of $T^0_0$ note that it is $\bar{\rho}_{kessence} [1+\delta \rho/\bar{\rho} ]$ and since in Gevolution it is multiplied to $-a^3$ and since critical density at redshift zero is 1 so we have\\
\be
3 a M_{pl}^2   \mathcal{H}^2\Omega = a^3   \frac{\bar{\rho}_D}{\rho_{crit} ^{0}=1} = a^3   \frac{\bar{\rho}^0_D a ^{-3(1+w)}}{\rho_{crit} ^{0}}= \Omega^{0}_{kess} a^{-3w}
\ee
\begin{empheq}[box=\mymath]{equation}
\begin{align}
 & T_0^0 (Gev)=  \Omega^0_{kess} a^{-3 w}  \Bigg[1+ \frac{1+w}{c_s^2} \Big(- 3 \mathcal{H}c_s^2 \pi- \Psi+   {\color{blue} ({\pi'}+ \mathcal{H} \pi) }  -  \Big(1-2 c_s^2 \Big) 
 \frac{(\vec{\nabla} \pi)^2}{2} \Big )   \Bigg ]
\nonumber \\ &
T^{i}_{0}(Gev)= - \Omega^0_{kess} a^{-3 w} (1+w) \Big[1 - (\frac{1}{c_s^2} -1)  \frac{(\vec{\nabla} \pi)^2}{2}  \Big ] \partial _i \pi 
\nonumber \\ &
T_{j}^{i}(Gev)= w  \, \Omega^0_{kess} a^{-3 w} \Bigg ( 1+  \frac{1+w}{w}\Big [ -3 \mathcal{H} w \pi- \Psi +   {\color{blue} ({\pi'}+ \mathcal{H} \pi) } -  \frac{(\vec{\nabla} \pi)^2}{2}   \Big] \delta_{j}^{i}  + \frac{1+w}{w} \delta^{i k} \partial_k \pi \partial_j \pi  \Bigg) 
\end{align}
\end{empheq}
{\color{red} What is $\delta P/\delta\rho$ here and is it comparable with other papers?}
In Gevolution we extract $\delta T_0^0/ \bar{T}_0^0$ which scales out the coefficient $\Omega^0_{kess} a^{-3 w} $. \\
The field equation is:
we take $d \tau=\tau_{n+1}-\tau_n $ and $x_{i,j,k} $ as lattice point. We solve the differential equation numerically as following;
\be
\pi_v= {\pi}'
\ee
\be
\pi^{n}= \pi ^{n-1}+\pi_v ^{n-\frac{1}{2}} d \tau
\ee
\be \label{eq3}
\pi_v ^{n+\frac{1}{2}}=\pi_v ^{n-\frac{1}{2}} + {\pi''} ^{(n)}  d \tau
\ee

We define the laplacian in code as following,
\begin{align}
& \nabla^2 \pi =-\frac{\pi^{n}_{i-1,j,k}+\pi^{n}_{i+1,j,k} +\pi^{n}_{i,j-1,k} +\pi^{n}_{i,j+1,k}+\pi^{n}_{i,j,k-1}+\pi^{n}_{i,j,k+1} -6 \pi^{n}_{i,j,k}  }{ a^2 dx^2}  
\end{align}
Moreover in order to get scalar in the vertices the derivatives like $\nabla \pi . \nabla \pi $ should be defined symmetric .
So we can rewrite the equation \ref{eq3} as below;
After some simplification we get,
\begin{align} 
%
 &\pi_v ^{n+\frac{1}{2}}= \frac{1}{1+ d\tau   \mathcal{H}^{(n)}  (1-3w) /2 - d\tau (1-c_s^2) \nabla^2 \pi^{(n)}/2} \times \Bigg[ \pi_v ^{n-\frac{1}{2}} - d \tau \Big [(1- 3w)\mathcal{H}^{(n)}   \frac{\pi_{v \; {i,j,k}}^{n-\frac{1}{2}} }{2}
     \nonumber
     \\
      &
  -3 { c_s^2 \mathcal{H}^{(n)}}\Big( 1- \frac{w}{c_s^2} \Big )\Psi^{(n) }
 - \, \frac{{\Psi}^{(n)}-{\Psi}^{(n-1)} }{d \tau}
      - 3  c_s^2  \, \frac{{\Phi}^{(n)}-{\Phi}^{(n-1)} }{d \tau}    
   +\Big( 3\mathcal{H}^2 (c_s^2 -w) + \mathcal{H}' (1-3c_s^2) \Big)\pi^{(n)}  
             \nonumber
     \\
      &
       - c_s^2 {\nabla^2 \pi ^{(n)}}  
   %
   + (1-c_s^2)\Psi^{(n)} {\nabla^{2} \pi^{(n)}  }    
      %___________
    - 2 c_s^2  \Phi ^{(n)}  {\nabla^2 \pi^{(n)}}
          %___________
     + {3 c_s^2  \mathcal{H}^{(n)} (1+w) }\pi^{(n)} {\nabla^2 \pi^{(n)} }   
          \nonumber
     \\
      &
               %___________
     -  (1-c_s^2)
 \Big( \frac{\pi_{v \; {i,j,k}}^{n-\frac{1}{2}} }{2} +\mathcal{H}  \pi^{(n)}  \Big) {\nabla^2  \pi^{(n)}} 
               %___________
    - (2 c_s^2-1) {\nabla  \Psi^{(n)}  . \nabla \pi ^{(n)} }
            %___________
    + c_s^2 {\nabla  \Phi ^{(n)} . \nabla \pi^{(n)}  }  
                              \nonumber
     \\
       & 
            %___________
              +\frac{\mathcal{H}^{(n)}} {2 } \Big(2+3w+c_s^2  \Big) \,{\nabla  \pi^{(n)} . \nabla \pi^{(n)} }  
                      %___________                    
         -2(1-c_s^2) \Big( \frac{{ \nabla  ( \pi_{v  \; {i,j,k}}^{n+\frac{1}{2}} +\pi_{v \; {i,j,k}}^{n-\frac{1}{2}} ) }  } {2}  + \mathcal{H}  \nabla\pi^{(n)} \Big) 
    \Big] \Bigg]
\end{align}
Since we have $\nabla \pi_v $ in the equation we use predictor corrector method as following,\\
In the first step we predict that the term $\nabla \pi \nabla \pi_v$ is small so we approximate $\nabla \pi_v ^{(n)}$ with $\nabla \pi_v^{n-\frac{1}{2}}$ then we calculate $\pi_v^{n+\frac{1}{2}}$ according to the formula with the guess, then we use the new $\pi_v^{n+\frac{1}{2}}$ into the full equations to correct $\pi_v^{n+\frac{1}{2}}$ . \\
 We have taken $\pi_{v  \; {i,j,k}}^{n} =\frac{(\pi_{v  \; {i,j,k}}^{n+\frac{1}{2}} +\pi_{v \; {i,j,k}}^{n-\frac{1}{2}} )}{2} $. Then we need to calculate $\mathcal{H}'$, ${\Psi}'$ and  ${\Phi}'$ in each loop, to calculate ${\Psi}'$ we save two $\Psi$ in each loop. \\
% \begin{align} 
% &\pi'' - \mathcal{H} \Big (1+ 3w \Big)\pi' -3 {a c_s^2 \mathcal{H}}\Big( 1- \frac{w}{c_s^2} \Big )\Psi -a \, {\Psi'}- 3 c_s^2 a \,{\Phi'} 
%  +3  c_s^2 \Big({-\mathcal{H}^2 + \mathcal{H}'} \Big) \pi 
% - c_s^2 {\nabla^2 \pi }
%     + (1-c_s^2)\pi {\nabla^2 \Psi }
%      - 2 c_s^2 \pi {\nabla^2 \Phi }
%            \nonumber
%   \\
%    &
%      + 3 c_s^2  H (1+w)\pi {\nabla^2 \pi }   
%  -  (1-c_s^2)
%   \pi \frac{\nabla^2\pi ' }{a}   
%   - (2 c_s^2-1) {\nabla  \Psi . \nabla \pi }
% + c_s^2 {\nabla  \Phi . \nabla \pi }  
%   +\frac{\mathcal{H}} {2 a } \Big(2+3w+c_s^2  \Big) \,{\nabla  \pi . \nabla \pi }     =0 
%%  -(\frac{1}{c_s^2}-1) \nabla^2 \Psi \pi+ 2 \nabla^2 \Phi \pi - 3 H (1+w) \pi \nabla^2 \pi  + (\frac{1}{c_s^2}-1) \pi \nabla^2 \dot{\pi}   \nonumber \\ &+ (2-\frac{1}{c_s^2})\nabla \Psi \nabla \pi - \nabla \Phi \nabla \pi -\frac{H} {2 c_s^2} \Big(2+3w+c_s^2  \Big) \nabla \pi \nabla \pi =0
%  \end{align} 
\section{Equation for $\Phi'$} in Class and Rees-Sciama effect.
We want to add a new equation for $\Phi'$ since Gevolution does not give the right solution in high k, which is maybe because of noises ?!..\\
We add the below equation which is $0i$ Einstein equation in conformal Newtonian gauge, eq 23.b of \url{https://arxiv.org/pdf/astro-ph/9506072.pdf}
\be
\Phi'=-\mathcal{H} \Psi+ \sum _i\frac{4 \pi G a^2 (\bar{\rho_i} + \bar{P_i}) \theta_i}{k^2}
\ee
How to implement it in Gevolution?!\\
We have,
\be
 (\bar{\rho} + \bar{P}) \theta =  \nabla_i T_0^i = i k^j T_j^0
\ee
But again we need the derivative of $T_0^i$ which is the same issue as $\Phi'$ instead we use $(00)$ Einstein equation, eq. 23a of \url{https://arxiv.org/pdf/astro-ph/9506072.pdf}
\be
\Phi'=-\mathcal{H} \Psi - \sum _i\frac{3 \delta_i} {2}
\ee
Actually in matter dominated universe the first order term vanishes, so class and Gevolution mismatch even in high redshift! so we need to check $\dot{\Psi}$ in Gevolution compared with second order perturbation theory.
\subsection{Rees-Sciama effect}
In matter dominated universe ${\Phi'}$ vanishes in linear order. Next order contribution would be,
\be
{\Phi'} = -\frac{3 H_0^2 }{2 k^2} {a'} \delta_2
\ee
where $\delta= a \delta_1 +a^2 \delta_2 $.
\be
\delta_2 (\vec{k}) = \int d^3 {q_1} \int d^3{q_2 } \;  \delta_D(\vec{k}-\vec{q_1}-\vec{q_2}) \;  F_2(\vec{q}_1 , \vec{q}_2)  \; \delta_1 (|\vec{q}_1|) \, \delta_1(|\vec{q}_2|)
\ee
\be
F_2(\vec{q}_1,\vec{q}_2)= \frac{5}{7} + \frac{1}{2} \frac{\vec{q}_1 . \vec{q}_2}{q_1 q_2} \Big ( \frac{q_1}{q_2} + \frac{q_2}{q_1} \Big) + \frac{2}{7} \frac{(\vec{q}_1 .\vec{q}_2)}{q_1^2 q_2^2}
\ee
So we obtain,
\be
P_{{\Phi'} }=  \frac{9}{4} (\frac{H_0}{k} )^4 {a'}^2 \; P_{22}
\ee
\be
P_{22} (k) = \int d^3 {q} P_{\delta} ({q}) P_{\delta} (|\vec{k}-\vec{q}|) \; F_2^2(\vec{q} , \vec{k} - \vec{q})
\ee
We  use the Growth factor,
\be
D^{+}= H(a) \frac{5 \Omega_m}{2} \int \frac{d \,a}{a^3 H(a)}
\ee
and physical Hubble ,
\be
H(a)=\sqrt{\Omega_m a^{-3} + (1-\Omega_m- \Omega_{\Lambda}-\Omega_{kess}) a^{-2}+ \Omega_{kess}^{-3(1+w)}+\Omega_{\Lambda}}
\ee
To make dimensionless quantity we have:
\be
P_{{\frac{\Phi'}{\mathcal{H}}} }=  \frac{9}{4} (\frac{H_0}{k} )^4 {a}^2 \; P_{22}
\ee
where $[  P_{{\frac{\Phi'}{\mathcal{H}}} }]=[P_{22}]= L^3$ and to make dimensionless powerspectrum we have $\mathcal{P}_{{\frac{\Phi'}{\mathcal{H}}} } = 2 \pi^2 k^3 P_{{\frac{\Phi'}{\mathcal{H}}} } $
\subsection{Initial condition}
-Do everything carefully and have some behaviours and approximations in mind of functions \\
-Reach something consistent\\
-
\subsection{gevolution tests}
-Compare with hi-class in different redshift\\
-Do check Stress tensor implemented alone,\\
-Invent some tests \\
-Put figure of Background and zero order modes of the field and Lorenzo's function


%%%%%%%%%%%%%%%%%
\section{Conclusions}
\setcounter{equation}{0}
%%%%%%%%%%%%%%%%%%%%%

 
\section*{Acknowledgements}

We acknowledge financial support from the Swiss National Science Foundation.


%%%%%%%%%%%%%%%%%%%%%%%%%%%%%%%
%%%%%%%%%%%%%%%%%%%%%%%%%%%%%%%
\appendix
%%%%%%%%%%%%%%%%%%%%%%%%%%%%%%%%%%%

%\bea
%&& \hspace*{-1cm}\l F_3^*(\bell_1,z_1)\DH(\bell_2,z_2)\re =   ~ \de(\bell_1-\bell_2) \times \nonumber \\ && \left\{\begin{array}{ll}
% 0 & \quad \mbox{ if } z_2>z_1 \\ 
%2\Om_m\frac{(r_1-r_2)r_2H^2_0}{r_1H(z_2)}(1+z_2)C_{\ell_2}^{\de_m\DH}(z_2,z_2)R^{\phi\De}(z_2,z_1)  &  \quad\mbox{ if } z_2<z_1
%\end{array}\right.
%\eea
This is the result which we have inserted in \eqref{e:Cellens}.
%
\bibliographystyle{JHEP}
\bibliography{biblio}
\end{document}






\begin{thebibliography}{999}
\newcommand{\bb}{\bibitem}


\bibitem{RuthBook} R. Durrer, {\it The Cosmic Microwave Background}. Cambridge University Press, 2008.


\bibitem{Lewis:2006fu} 
  A.~Lewis and A.~Challinor,
  ``Weak gravitational lensing of the cmb,''
  Phys.\ Rept.\  {\bf 429}, 1 (2006)
  [astro-ph/0601594].
  %%CITATION = ASTRO-PH/0601594;%%

\bibitem{Fanizza:2015swa} 
  G.~Fanizza, M.~Gasperini, G.~Marozzi and G.~Veneziano,
  ``A new approach to the propagation of light-like signals in perturbed cosmological backgrounds,''
  arXiv:1506.02003 [astro-ph.CO].
  %%CITATION = ARXIV:1506.02003;%%
     
\bibitem{DiDio:2014lka} 
  E.~Di Dio, R.~Durrer, G.~Marozzi and F.~Montanari,
  %``Galaxy number counts to second order and their bispectrum,''
  JCAP {\bf 1412}, 017 (2014)
  [arXiv:1407.0376 [astro-ph.CO]].
  %%CITATION = ARXIV:1407.0376;%%   
    
    
  

 \end{thebibliography}




